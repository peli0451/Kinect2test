\documentclass[
	11pt,
	xcolor={dvipsnames},
	hyperref={
		bookmarks=true,
		pdftex,
		bookmarksnumbered=true,
		pdfencoding=auto}
	]{beamer}
%
\usepackage[utf8]{inputenc}
\usepackage[T1]{fontenc}
\usepackage[ngerman]{babel}
\usepackage[babel,german=quotes]{csquotes}
\usepackage{ifthen}
\usepackage[backend=bibtex,style=numeric-comp]{biblatex}

\usepackage{tikz}

\definecolor{TUGreen}{HTML}{006666}
\definecolor{TUOrange}{HTML}{FF6600}
\definecolor{TUBlue}{HTML}{003366}
\definecolor{TURed}{HTML}{CC0000}
\definecolor{TULightblue}{HTML}{6699CC}
\definecolor{TUGreenL}{HTML}{E5F2F3}
\definecolor{TUOrangeL}{HTML}{FBE5D6}
\definecolor{TUBlueL}{HTML}{E3EFFA}
\definecolor{TURedL}{HTML}{FDE1E1}
\definecolor{TULightblueL}{HTML}{E3EFFA}

\definecolor{MetroRed}{HTML}{D11141}
\definecolor{MetroGreen}{HTML}{00B159}
\definecolor{MetroBlue}{HTML}{00AEDB}
\definecolor{MetroOrange}{HTML}{F37735}
\definecolor{MetroYellow}{HTML}{FFC425}

\useoutertheme{infolines}
\useinnertheme{rectangles}

\setbeamercolor{alerted text}{fg=TURed}
\setbeamercolor{block body}{bg=TUBlueL}
\setbeamercolor{block title}{fg=white,bg=TUBlue}
\setbeamercolor{block body alerted}{bg=TURedL}
\setbeamercolor{block title alerted}{fg=white,bg=TURed}
\setbeamercolor{block body example}{bg=TUGreenL}
\setbeamercolor{block title example}{fg=white,bg=TUGreen}
\setbeamercolor{frametitle}{fg=black,bg=Black!20!White}
\setbeamercolor{item}{fg=TUBlue}
\setbeamercolor{structure}{fg=Black!70!White}
\setbeamercolor{title}{fg=TUBlue,bg=Black!20!White}

\setbeamerfont{description item}{series=\bfseries}

\beamertemplatenavigationsymbolsempty

\newcommand*{\header}{\printsec}

\setbeamertemplate{headline}{
	\vskip2ex\color{gray}{\par\centering\strut\header\par}\vskip1ex
}
\setbeamertemplate{footline}[frame number]

\newcommand{\printsec}{\thesection\hspace{1em}\insertsectionhead}
\newcommand{\printsubsec}{\large\thesection.\thesubsection\hspace{1em}\insertsubsectionhead}

\usepackage{appendixnumberbeamer}

\usepackage{pifont}	
\newcommand{\cmark}{\ding{51}}
\newcommand{\xmark}{\ding{55}}

\newcommand<>*{\hl}[2][TURed]{%
	\begingroup%
	\textcolor#3{#1}{\bfseries #2}%
	\endgroup%
}

\newcommand{\printtocframe}{%
\begingroup
	\let\header\relax
	\begin{frame}{Vortragsstruktur}
		\tableofcontents
	\end{frame}%
\endgroup
}

\newcommand{\printcurrenttocframe}{%
\begingroup
	\let\header\relax
	\begin{frame}{\printsec}
		\tableofcontents[currentsection]
	\end{frame}%
\endgroup
}
%
\title{Gestensteuerung einer 3D-Anwendung mittels Kinect}
\subtitle{Projektseminar}
\author[Mario Janke, Peter Lindner, Patrick Stäblein]{Mario Janke\\Peter Lindner\\Patrick Stäblein}
\institute{%
	\color{gray}%
	\normalsize 	Technische Universit\"at Ilmenau\\%
	\footnotesize	Fakult\"at f\"ur Informatik und Automatisierung\\%
	\scriptsize 	Institut f\"ur Praktische Informatik und Medieninformatik\\%
	\scriptsize 	Fachgebiet Graphische Datenverarbeitung%
}
\date{21. Juli 2017}
%
%
\begin{document}

\begin{frame}[plain]
	\maketitle
	\begin{tikzpicture}[remember picture,overlay]
		\node[anchor=south east,outer sep=.2cm] at (current page.south east) {\includegraphics[scale=.5]{../pictures/logo.jpg}};
	\end{tikzpicture}
\end{frame}

\section{Aufgabenstellung}
\begin{frame}{\header}
\begin{itemize}
	\item 3D-Anwendung gegeben\\
		\qquad$\rightsquigarrow$ \textbf{Gegeben:} Präsentation einer 3D-Szene\\
	\item Bewegen, Rotieren und Objekte handhaben\\
		\qquad$\rightsquigarrow$ \textbf{Aufgabe \#1:} Gestensteuerung
	\item Anwendungsszenario\\
		\qquad$\rightsquigarrow$ \textbf{Aufgabe \#2:} Mastererkennung
	\item Umsetzung\\
		\qquad$\rightsquigarrow$ \textbf{Trackingsystem:} Microsoft Kinect
\end{itemize}
\end{frame}

\printtocframe

\section{Trackingsystem}
\begin{frame}{\header}
\begin{itemize}
	\item Nutzung der Microsoft Kinect\\
		\qquad$\rightsquigarrow$ Ermittlung von Tiefendaten durch Infrarotprojektion
	\item Tracking von bis zu 6 Personen gleichzeitig möglich
	\item KinectSDK\\
		\qquad$\rightsquigarrow$ Liefert Skelettkoordinaten + Handstates
		
\end{itemize}
\end{frame}

\section{Entwurfsentscheidungen}
\begin{frame}{\header}
\begin{itemize}
	\item Kennenlernphase
	\item Objektorientierung
	\item State-Machine-Pattern
	\item eigene Gestendefinition
\end{itemize}
\end{frame}
\section{Gestensteuerung}
\section{Mastererkennung}
\section{Bewertung \& Ausblick}
\begin{frame}{\header}
\begin{itemize}
\item Kinect bedingt geeignet
\item Probleme (Ausschnitt):
	\begin{itemize}
		\item Trackingfehler der Skelette (Skelettänderungen)
		\item falsche Proportionen
		\item Abhängigkeit von Verdeckung und Verdrehung
		\item Beleuchtungsabhängigkeit
		\item Jitter
		\item z.\,T. werden schlechte Daten nicht als solche erkannt
	\end{itemize}
\item Pro: Preis-Leistungs-Verhältnis
\end{itemize}
\end{frame}

\begin{frame}{Mögliche Verbesserungen}
\begin{itemize}
\item Steuerung zufriedenstellend, ggf. Anpassung an abweichende Zwecke
\item weitere Schnittstellen nach außen, konfigurierbare Parameter
\item Erweiterung der Steuerung
\item Mastererkennung noch robuster gestalten, ggf. über volles Bewegungsprofil (Trade-Off: Aufwand $\rightsquigarrow$ Ansätze meist abhängig vom Einsatzszenario)
\item groß angelegte Tests
\end{itemize}
\end{frame}

\end{document}