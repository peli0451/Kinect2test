\documentclass[12pt,a4paper]{article}
\usepackage[utf8]{inputenc}
\usepackage[T1]{fontenc}
\usepackage[ngerman]{babel}
\usepackage{amsmath}
\usepackage{amsfonts}
\usepackage{amssymb}
\usepackage{lmodern}
\usepackage{tikz-uml}
\usepackage[singlelinecheck=off, list=off]{caption}
\usepackage{enumitem}
\usepackage[backend=bibtex]{biblatex} 
\usepackage{float}
\makeatletter
\def\blx@maxline{77}
\makeatother
\addbibresource{literatur.bib}
\usepackage[colorlinks,bookmarksopen]{hyperref}

\newcommand*{\Title}{Projektseminar: Gestensteuerung einer 3D-Anwendung mittels Kinect}
\newcommand{\Authors}{Mario Janke,\\Peter Lindner,\\Patrick Stäblein}
\title{\Title}
\author{
	Mario Janke\\
	Peter Lindner\\
	Patrick Stäblein}
\date{}

\usepackage{fancyhdr}
\usepackage{vmargin}
\usepackage{setspace}
\onehalfspacing
\pagestyle{fancy}  %lustige fuß- und kopfzeilen
\parindent 0pt	
\parskip 0pt %
\emergencystretch 20pt
\setmarginsrb{3.0cm}{2.5cm}{2.5cm}{1.5cm}%  %ränder links oben rechts unten
             {1.0cm}{1.5cm}{1.0cm}{1.7cm}%  %kopf und fußzeile, höhe, abstand
\setlength{\leftmargin}{3.0cm}
\setlength{\textwidth}{15.5cm}
\setlength{\headwidth}{15.5cm}
\setlength{\topmargin}{2.0cm}
\lfoot{\footnotesize{\emph{\Title}}}
\cfoot{}
\rfoot{\footnotesize{\emph{\thepage}}}
\rhead{\emph{\leftmark}}
\chead{}
\lhead{}
\renewcommand{\footrulewidth}{0.5pt}
\renewcommand{\headrulewidth}{0.5pt}
\usepackage{subcaption}
\widowpenalty10000
\clubpenalty10000
\begin{document}
\begin{titlepage}
\setmarginsrb{2.5cm}{2.5cm}{2.5cm}{2.5cm}%  %ränder links oben rechts unten
             {1.0cm}{1.0cm}{1.0cm}{1.7cm}%  %kopf und fußzeile, höhe, abstand
\begin{center}
\vspace*{-2cm}
\includegraphics[width=.4\textwidth]{pictures/logo.jpg}\par 
\vspace*{1cm}{\small
Technische Universität Ilmenau\\
Fakultät für Informatik und Automatisierung\\
Institut für Praktische Informatik und Medieninformatik\\
Fachgebiet Graphische Datenverarbeitung}\par
\vfill
{\large Projektseminar}\par
\vspace*{.5cm}
%\rule{.9\textwidth}{4pt}\\[4ex]
{\huge\bfseries Gestensteuerung einer 3D-Anwendung mittels Kinect}\\[2ex]
%\rule{.9\textwidth}{4pt}\par
\vfill
\begin{tabular}{r l}
\emph{Autoren:}	& Mario Janke\\
	& Peter Lindner\\
	& Patrick Stäblein\\
&\\
\emph{Betreuer:} & M.\,Sc. Julian Meder
\end{tabular}\par
\vfill
{\today}\par
\end{center}
\end{titlepage}
\tableofcontents
\clearpage
\section{Rahmenbedingungen}
Die vorliegende Ausarbeitung entstand im Rahmen des Projektseminars im Master-Studiengang Informatik an der Technischen Universität Ilmenau. Ziel und Zweck des Projektseminars ist die teambasierte Auseinandersetzung mit einem Forschungsgegenstand unter Verwendung von Fachliteratur.\par
Im Folgenden werden die Grundgegebenheiten erläutert, die genaue Aufgabenstellung formuliert und das titelgebende Tracking-System Microsoft Kinect eingeführt.
	\subsection{Grundlagen \& Technik}
	Gegeben ist eine bereits vorhandene 3D-Anwendung, die zu Demonstrationszwecken genutzt wird. Innerhalb der Anwendung ist es möglich,
	\begin{itemize}
		\item Objekte zu laden und damit anzeigen zu lassen sowie
		\item die Kamera (bzw. Kameras) zu manipulieren, d.\,h. zu bewegen, zu rotieren und zu zoomen,
	\end{itemize}
zusätzlich geplant ist später
	\begin{itemize}
		\item geladene Objekte manipulieren, in diesem Falle skalieren oder löschen zu können.
	\end{itemize}
	Das Programm rendert dabei zwei Ausgabefenster, in denen die Szene dargestellt ist, wobei die Kameras einen 3D-Aufbau bilden.\par
	Die so beschriebene Ausgabe wird über zwei Projektoren von hinten auf eine Projektionsfläche geworfen -- ein Projektor für die linke Kamera und einer für die rechte. Wird die \glqq Leinwand\grqq{} von vorne durch eine Shutterbrille betrachtet, entsteht der 3D-Eindruck.\par 
	Die Steuerung der Anwendung erfolgt über Tastatur und Maus bzw. Präsentationspointer.
	\subsection{Aufgabenstellung}
	Ziel des Projektseminars ist es, die Steuerung der Anwendung hinsichtlich einer Präsentation vor einer Zuschauergruppe zu erleichtern und intuitiv zu gestalten, sodass parallel an der Universität vorhandene (und bislang ungenutzte) Technik verwendet und präsentiert werden kann. In diesem Sinne geeignet und vorgeschlagen sind
	\begin{itemize}
		\item ein professionelles Trackingsystem zum Tracken von Raumpunkten und
		\item die Verwendung einer Microsoft Kinect 2 zur Gestenerkennung.
	\end{itemize}\par 
	Das damit entwickelte Programm soll Folgendes leisten:
	\begin{itemize}
		\item Es soll in der Lage zu sein, sämtliche Steuerung und Manipulation, die oben beschrieben wurde durchzuführen.
		\item Die Bedienung soll sehr intuitiv und einfach sein, d.\,h. etwaige Gesten müssen bezüglich der ihnen zugeordneten Aktion einleuchtend und leicht auszuführen sein.
		\item Das Programm soll möglichst einfach eingebunden und wiederverwendet werden können.		
	\end{itemize}
		\subsection{Aufgabenstellung}\label{sec:aufg}
	Aus der oben angeführten Motivation heraus, ist es das Ziel unseres Projektes, die Steuerung der uns gegebenen Anwendung hinsichtlich ihrer Präsentation vor einer Zuschauergruppe zu erleichtern und intuitiv zu gestalten. Dabei wird besonderer Wert auf das Eintauchen des vorführenden Masters in die 3D-Szene gelegt. Diese Anforderung soll durch die Implementierung einer Gestensteuerung erfüllt werden. Parallel soll der zusätzliche Zweck erfüllt werden, weitere am Fachgebiet vorhandene, bislang jedoch ungenutzte Technik zu verwenden und präsentieren zu können. Dies betrifft das Trackingsystem, mit welchem die genannte Gestensteuerung umgesetzt werden soll. Hierbei standen zur Auswahl:
	\begin{itemize}
		\item ein professionelles Trackingsystem zum Tracken von Raumpunkten und
		\item die Verwendung einer Microsoft Kinect 2 zur Gestenerkennung.
	\end{itemize}
	Das genannte professionelle Trackingsystem ist ein Motion-Capture-System, das über angebrachte Marker und Wands funktioniert. Die Kinect hingegen verwendet keine am Körper angebrachten Merkmale und leitet die 3D-Information aus ihrer Sicht auf ein projiziertes Infrarotmuster ab.\par 
	Bereits sehr früh (so früh, um sie als verfeinerte Aufgabenstellung betrachten zu können) fiel die Entscheidung, die Kinect als Trackingsystem zu verwenden. Dies hatte vielerlei Gründe. Zum einen ist sie mit ihrer Herkunft aus der Spieleindustrie einem breiten Publikum bekannt und daher gegebenenfalls besser zur Präsentation geeignet. Hier kommt ebenfalls zugute, dass der Nutzer ohne jeglichen Aufwand mit dem Bedienen der Kinect beginnen kann, markerbasierte Trackingsysteme haben diesen \glqq{}Plug-and-Play\grqq{}-Vorteil nicht. Andererseits ist es auch der Popularität der Kinect zu verdanken, dass es eine sehr gute Quellenlage im Internet gibt. Die Kinect verfügt über eine (wenn auch nicht in großem Maße ausführliche) Online-Dokumentation und da sie mit SDK und API veröffentlicht wurde finden sich in einem doch etwas breiteren Rahmen Lösungsdiskussionen und -präsentationen im Netz. Nicht zuletzt war sie allen Projektmitgliedern bekannt und weckte aufgrund ihrer Herkunft bereits Interesse.\par\medskip
	Das so ausgewählte Trackingsystem soll nun also zur Implementierung einer Gestensteuerung der gegebenen Anwendung genutzt werden. Das dafür entwickelte Programm soll Folgendes leisten:
	\begin{itemize}
		\item Es soll in der Lage zu sein, sämtliche Steuerung und Manipulation, die oben beschrieben wurde durchzuführen, d.\,h. Kamera und Objekte steuern können.
		\item Die Bedienung soll sehr intuitiv und einfach sein, d.\,h. etwaige Gesten müssen bezüglich der ihnen zugeordneten Aktion einleuchtend und leicht auszuführen sein.
		\item Die Steuerung soll ihrem Zweck angemessen genau sein, am besten ist hier eine glatte 1-zu-1-Übertragung von Handbewegungen auf die Szene.
		\item Das Programm soll möglichst einfach eingebunden und wiederverwendet werden können.		
	\end{itemize}
	Aus dem Anwendungszweck des Originalprogrammes heraus erwuchs die zusätzliche Anforderung, eine Mastererkennung / -verwaltung zu implementieren, das heißt ein Verfahren, das garantiert, dass auch nur die dafür vorgesehene Person das Programm steuert und niemand sonst. Im Rahmen der genannten öffentlichen Präsentation muss ausgeschlossen sein, dass ein Fremder die Kontrolle über das Programm gewinnen kann und die Vorführung dadurch -- gewollt oder unbewusst -- behindert. Zusätzlich soll die ausgezeichnete Person auch später wieder erkannt werden, nachdem sie etwa einen Augenblick lang nicht im von der Kamera abgedeckten Bereich war.\par\bigskip
	Damit ist die vollständige Liste der Anforderungen gegeben und wird hier zur Übersicht nochmals in ihrem Kern zusammengefasst:\par 
	Ziel ist die Entwicklung einer Software
	\begin{itemize}
	\item unter Verwendung vorhandener Technik (gemeint ist das Trackingsystem Kinect),
	\item die eine Gestensteuerung der gegebenen Anwendung ermöglicht und
	\item dabei nur einer ausgezeichneten Person diese Steuerung erlaubt.
	\end{itemize}
	\subsection{Eigenschaften der Kinect} 
\begin{figure}[htbp] 
  \centering
     \includegraphics[width=\textwidth]{pictures/kinectskeleton-map2.png}
     
  \caption{Gelenkpunkte die mit der Kinect getrackt werden können}
  \footnotesize Quelle: \cite{tracking2}
  \label{fig:Bild1}
\end{figure}

	Wir stellen in diesem Abschnitt nur die für uns interessanten Eigenschaften und Möglichkeiten der Kinect vor (hinsichtlich unserer Aufgabe und der Rahmenbedingungen). Die Kinect erkennt visuell den 3D-Raum vor sich. Dabei werden Personen als solche detektiert und konfidenzbasiert mit einem primitiven und grobgranularen Skelett ausgestattet. Hierbei lassen sich die Koordinaten der oben abgebildeten Gelenkpunkte mit Hilfe des Kinect SDKs abfragen. Die praktische Reichweite für die Erkennung einer Person beträgt etwa 1,2 bis 3,5 Meter. Dieses Tracking ist für bis zu sechs Personen zeitgleich möglich. Weiterhin wird für beide Hände einer getrackten Person ein \glqq Handzustand\grqq~erkannt, nämlich ob die Hand offen oder geschlossen ist, oder die sogenannte Lassogeste gebildet wird (etwa nur zwei Finger ausgestreckt). Kann einer Hand keiner dieser Zustände zugeordnet werden, ist ihr Status unbekannt. Diese Daten (Skelett und Status pro getrackter Person) können unter Verwendung der USB-Schnittstelle und des Kinect-SDKs abgegriffen werden. Sie werden dafür 30 mal in der Sekunde zur Verfügung gestellt.
	%
	%
\clearpage
\section{Vorüberlegungen}
		Für unser Vorgehen zentral sind die folgenden beiden Bereiche:
	\begin{enumerate}
		\item die technische Umsetzung, d.\,h.
		\begin{itemize}
		\item das korrekte Erkennen und Werten von Gesten einer ausgezeichneten getrackten Person
		\item das korrekte Berechnen notwendiger Bewegungsparameter
		\item die Einbindung in die bestehende Applikation
		\end{itemize}
		\item die Interaktion mit dem Benutzer, d.\,h.
		\begin{itemize}
		\item das Entwerfen intuitiver und eingängiger Gesten für die verschiedenen Zwecke
		\item das Auszeichnen einer getrackten Person als \glqq Master\grqq, der das Programm steuert
		\end{itemize}
	\end{enumerate}
	Wir stellen in diesem Abschnitt die zentralen unmittelbaren Beobachtungen vor, die sich aus der Aufgabenstellung und dem Versuchsaufbau ziehen lassen.\par\bigskip
	Ausgehend von der Aufgabenstellung kann man abstrahierend zwischen zwei primitiven Steuerungsmodi unterscheiden:
	\begin{itemize}
	\item einem Modus, in dem die Kamera verschoben und rotiert werden kann \&
	\item einem Modus, in welchem Objektmanipulationen möglich sind.
	\end{itemize}
	Der Benutzer sollte sich zu jedem Zeitpunkt nur in maximal einem dieser Modi aufhalten, d.\,h. gleichzeitige Kamera- und Objektmanipulation wird ausgeschlossen. Diese Vereinfachung treffen wir, da damit weniger komplexe Gesten benötigt werden und eine solche simultane Manipulation keine praktische Relevanz besitzt. Für Manipulationen, die man sowohl für die Kamera, als auch für Objekte haben will, bietet dies zudem eine geeignete Kapselung, da z.\,B. Rotationsparameter berechnet werden und dann nur entschieden werden muss, ob sie auf die Kamera oder ein Objekt angewendet werden, je nach Modus. Dies reduziert die Gesamtzahl nötiger Gesten.\par\medskip
	Die Kinect ermöglicht ein Tracking des gesamten Körpers für mehrere (genauer sechs) Personen. Wir beschränken uns aus naheliegenden Gründen jedoch auf einen Teil dieses Spektrums:
	\begin{itemize}
		\item Wir benötigen nur eine Person, die die Anwendung (möglichst ungestört) steuert. Eine genauere Auswertung der restlichen Personen, ihrer Skelette etc. ist unnötig.
		\item Die in unserem Anwendungsfall intuitiven Gesten werden ausschließlich mit den Händen (bzw. Armen) durchgeführt.
	\end{itemize}
	Primitive Erkennungsmöglichkeiten eines Masters kann man etwa aus der Entfernung der getrackten Personen zur Kamera und der Position der Personen im Raum gewinnen. Genauere Erklärungen folgen weiter unten.\par 
	Intuitive Gesten für Verschiebungen imitieren das Verschieben eines großen Gegenstands, etwa einer imaginären Box, sodass hier etwa ein Verschieben der flachen Hand in der Luft naheliegt. Für eine intuitive Drehgeste eignet sich die Vorstellung eines imaginären Lenkrads, genauer gesagt einer Lenkkugel, bei der die Rotation um eine Raumachse nach dem Lenkradprinzip erfolgt. Eine intuitive Geste zur Objektauswahl ist offenbar eine Greifgeste.
	\subsection{Priorisierung der Aufgaben}
	%TODO
	\subsection{Aufgabenverteilung im Team}
	Das Projekt wurde zu viert begonnen, wobei einer der Teilnehmer gleich zu Beginn wieder absprang. Die Aufgabenverteilung im Team wurde ab der Einarbeitungsphase mit dem Kinect-System dynamisch vorgenommen. So kristallisierten sich im Laufe der Zeit diverse Zuständigkeitsbereiche heraus, die grob so umrissen werden können:
	\begin{description}
		\item[Mario Janke] kümmerte sich vor allem um die Mathematik im Hintergrund, präziser um die diversen Berechnungen von Parametern aus den vorliegenden Kinect-Daten. Als klar wurde, dass das Projekt aufgrund der Eigenschaften der Kinect zusätzliche Robustheitsmechanismen benötigte, wurde das Management unserer Puffer zusätzlicher Aufgabenbereich.
		\item[Peter Lindner] sorgte vorderrangig für die Strukturierung des Codes. Dies erstreckt sich auf die Umsetzung des objektorientierten Programmierparadigmas mit der Ausarbeitung und Erstellung der Klassenhierarchie. Die im Zuge dessen entstandene Zustandsmaschine wurde in der Folge von ihm verwaltet. Da für deren Existenz zu großen Teilen das beabsichtigte Wirkungsprinzip der Gesten war, schloss sich dem Aufgabenbereich das Gestenmanagement an.
		\item[Patrick Stäblein] war für das Ansammeln grundlegender Informationen zur Kinect -- gerade in der Anfangsphase -- verantwortlich und programmierte einen Großteil der Mastererkennungsmechanismen in der zweiten Phase des Projekts.
	\end{description}
	Parallel zur Arbeit am Programmcode entstand das vorliegende Dokument und später die Vorbereitung der Abschlusspräsentation. Dabei orientierten sich die vom jeweiligen Projektmitglied bearbeiteten Themengebieten an ihren Zuständigkeiten und damit Wissenschwerpunkten aus der Programmierung.
	\subsection{Werkzeuge}
	In diesem Abschnitt sollen die Tools genannt und erklärt werden, die für die Entwicklung unserer Software vordergründig waren.\par
	Das im Zentrum stehende Trackingsystem \emph{Microsoft Kinect Version 2} war durch das Fachgebiet gegeben. Zur Arbeit damit stand ein Raum bereit, der über die Kinect und einen 3D-Kamera-Aufbau zur Projektion verfügte. Ferner durfte die Kinect für Heimtests auch ausgeliehen werden.\par 
	Mit dem Kinect SDK wird eine Softwarelösung namens \emph{Kinect Studio} ausgeliefert, die sich bei der Kinect-Programmierung zum visuellen Debugging eignet. Im Kinect Studio können die verschiedenen Sensoraufnahmen der Kinect nebst der interpretierten Skelette und Hand-States sowie der festgestellten Tiefensituation betrachtet werden. Diese Features können getoggelt oder umgeschaltet werden. Die Tiefenkarte wird per Falschfarbendarstellung und, sofern erwünscht, sogar dreidimensional präsentiert. Das Kinect Studio erwies sich als sehr hilfreich, um die Güte der Kinect-Daten zu überprüfen und zu erkennen, ob die Kinect einen Teil des Aufnahmebereiches fehlinterpretiert. Dies war zum Teil notwendig, um bei der Programmierung schnell und einfach zwischen Fehlern des Programmes und fehlerhaften Kinect-Daten unterscheiden zu können.\par 
	Ebenfalls zum Kinect-SDK gehörig ist der sogenannte \emph{Visual Gesture Builder}, mit dem Gesten(folgen) aufgenommen und in Programme eingespeist werden können. Da wir uns (s.\,u.) für einen anderen Weg der Gestenimplementierung entschieden haben, fanden hiermit nur kleinere Tests in der Anfangsphase statt.\par 
	Die Kinect-API kann mit JavaScript, C++ oder C\# verwendet werden. Da das uns im Rahmen der Aufgabenstellung übermittelte Programm, für das unsere Gestensteuerung gedacht ist, in C++ geschrieben war, verwendeten wir aus Kompatibilitäts- und Einheitlichkeitsgründen heraus ebenfalls C++ und entwickelten und debuggten mit dem \emph{Microsoft Visual Studio 2015} unter \emph{Microsoft Windows 10}.\par 
	Zur Versionsverwaltung nutzten wir das Kommandozeilentool git mit \href{http://github.com}{GitHub}.\par 
	%
	%
\clearpage
\section{Entwurfsentscheidungen}\label{sec:entwurf}
	\subsection{Der Master}
	Der Master ist die Person (unter den getrackten Personen), der es obliegt, die Anwendung zu steuern, d.\,h. in unserem Anwendungsfall der Präsentation ist der Master der Präsentierende.\par
	Es muss gewährleistet werden, dass nur der Master das Programm steuert und dabei von weiteren Personen im Raum nicht (bzw. nicht ohne weiteres) gestört werden kann. Die Erkennung muss robust gegen Jittering der Kinectdaten sein.
	\subsection{Gesten und ihre Wirkung}\label{sec:gesten}
Für die eingangs erwähnte Aufgabenstellung war es notwendig, bestimmte Programmfunktionalitäten mit Gesten zu verbinden. Einerseits hätte die Möglichkeit bestanden, mit dem Kinect-eigenen Visual Gesture Builder Gesten aufzunehmen und einzulernen. Diese Gesten werden dann als Datenbank ins Programm geladen und bei Vorführung erkannt. Dies erspart natürlich primitive aber umständliche Low-Level-Erkennungsmechanismen. Weiterhin sind hierdurch einige weiterführende Möglichkeiten gegeben wie etwa die Rückgabe, bis zu welchem Punkt eine Geste bereits ausgeführt wurde (in Bezug zur Gesamtgeste, d.\,h. beispielsweise wieviel Prozent einer Armbewegung vordefinierte Länge bereits ausgeführt wurde). Anderseits wiederum können durch die Kinect-Rohdaten auch eigene Erkennmechanismen implementiert werden. Dies bietet dem Programmierer die vollständige Kontrolle und Freiheit darüber, wie er Gesten definiert und auswertet, statt wie im Falle des Gesture Builders auf ein gewisses Rahmenwerk angewiesen zu sein. Änderungen können kurzfristig und schnell vorgenommen werden und für einfache Projekte ist die Zusatzfunktionalität, die der Visual Gesture Builder gestattet nicht vonnöten, der eher für komplexere Gestenfolgen ausgelegt zu sein scheint. Demgegenüber ist für diese Direktimplementierung von Gesten aber die bereits erwähnte Low-Level-Erkennung zu implementieren, d.\,h. ein Extrahieren von Bewegungen und Bewegungsrichtungen aus den Skelett- und Gelenkdaten, die die Kinect bestimmt. Aus dieser Gegenüberstellung heraus, ist für den gegebenen Einsatzzweck eine direkte Gestenerkennung die sinnvollere Alternative. Mit dem Visual Gesture Builder ist es schlicht nicht möglich, eine wie beabsichtigte $1$"=zu"=$1$"=Abbildung zwischen Geste und Wirkung zu bewerkstelligen.\par 
	Auch bei der Low-Level-Erkennung gibt es jedoch verschiedene Ansätze bzw. Ausprägungen. Es ist sogar das Implementieren nicht ganz primitiver Gestenfolgen möglich, indem eine Geste zeitlich und räumlich in verschiedene Segmente unterteilt wird. Dies sei an einem Beispiel erläutert: Es soll eine Winkgeste der rechten Hand erkannt werden. Die Geste wird in zwei Segmente geteilt. Ein Wechsel zwischen den Segmenten findet statt, wenn die horizontale Position der Hand und des Ellenbogens wechseln. Wird dieser Übergang dreimal in Folge erkannt, so wurde die Winkgeste präsentiert.\par
	Eine genauere Auseinandersetzung mit der Aufgabenstellung und allgemeinen Vorstellungen von intuitiven Gesten für die zu realisierenden Funktionalitäten zeigte jedoch auf, dass auch eine Segmenteinteilung von Gesten für das Projekt nicht notwendig ist. Stattdessen sind die gegebenen Aufgaben (ein Verschieben oder Rotieren per Geste) in ihrer Struktur simpel genug, um die verschiedenen Wirkungen mit diskreten Gesten zu erzeugen, d.\,h. es genügt die Erkennung einer Geste durch bestimmte Zustände der Kinect-Rohdaten zu einem einzigen Zeitpunkt. Um die Wirkung jedoch zu erzielen, ist natürlich auch eine Betrachtung der Geste über mehrere Frames notwendig.\par\medskip
	Im Folgenden ist der im Projekt Verwendung findende Gestenkatalog erklärt. Dabei wird darauf eingegangen, was der Benutzer vorführen muss, damit die Geste erkannt wird und wie die Geste genutzt wird, um in der Anwendung die Kamera oder Objekte zu manipulieren:\par\bigskip
	\begin{description}
		\begin{figure}[H]
		\centering
		\includegraphics[width=.8\textwidth]{pictures/translate_.png}
		\caption{Die (Kamera-)Translations-Geste, mit und ohne eingezeichnetes Skelett und HandStates.}\label{fig:translateg}
		\end{figure}
		\item[TRANSLATE\_GESTURE] (siehe Abb. \ref{fig:translateg})\par
		Der Benutzer hat beide Hände geöffnet, mit den Handflächen zur Kamera (wichtig ist nur, dass die Kinect beide Hände als offen erkennt, die genaue Haltung ist dabei egal). Ein paralleles Verschieben der beiden Hände in eine Richtung bewirkt ein zur Bewegungsgeschwindigkeit proportionales Verschieben der Kamera in diese Richtung.\par 
		Wie in den Vorüberlegungen (Abschnitt \ref{sec:vor}) schon erwähnt ist dies eine einfache Übertragung des Prinzips nachdem eine Person in der Realität einen großen Gegenstand verschieben würde und wird daher der geforderten Intuitivität gerecht.
		\par
		\begin{figure}[H]
		\centering
		\includegraphics[width=.8\textwidth]{pictures/rotate_.png}
		\caption{Die (Kamera-)Rotations-Geste, mit und ohne eingezeichnetes Skelett und HandStates.}\label{fig:rotateg}
		\end{figure}		
		\item[ROTATE\_GESTURE] (siehe Abb. \ref{fig:rotateg})\par
		Der Benutzer hat beide Fäuste geballt. Dann bewirkt eine gleichzeitige Bewegung der Hände auf einer Kreisbahn eine Rotation der Kamera um die Senkrechte des zugehörigen Kreises.
		Analog zur Verschiebegeste fand diese Geste bereits in den Vorüberlegungen (Abschnitt \ref{sec:vor}) Erwähnung. Die Aufgabe, eine Rotation im Raum zu vollführen ist leider weniger alltäglich als die des Verschiebens von Gegenständen. Ein verwandtes Prinzip ist jedoch das des Lenkrades, das man sich --  ins Dreidimensionale erweitert -- als Lenkkugel vorstellen kann. Mittels der üblichen Lenkbewegung kann dann um beliebige Raumachsen rotiert werden. Ähnlich findet dies auch etwa bei der Bedienung von Münzfernrohren statt. Wegen dieser Analogien, kann davon ausgegangen werden, dass der Benutzer schnell ein intuitives Verständnis von der Wirkungsweise dieser Geste entwickelt.
		\par
		\begin{figure}[H]
		\centering
		\includegraphics[width=.8\textwidth]{pictures/grab_.png}
		\caption{Die Objektmanipulations-Geste, mit und ohne eingezeichnetes Skelett und HandStates.}\label{fig:grabg}
		\end{figure}
		\item[GRAB\_GESTURE] (siehe Abb. \ref{fig:grabg})\par
		Zunächst war angedacht, dass die Objektmanipulation dieselben Gesten verwendet wie die Kameramanipulation und die Unterscheidung, was manipuliert wird durch einen globalen Zustand gefällt wird. Bei näherer Betrachtung dieses Ansatzes und ersten Tests dessen fiel auf, dass es so schwierig ist, zwischen Kamera- und Objektmanipulation zu wechseln. Weiterhin schien es während des Testens weniger intuitiv als zuvor angenommen, ein Objekt auf diese Art und Weise zu manipulieren. Es bedarf hier also anderer Ansätze.\par 
		In das Problem der Objektmanipulation eingeschlossen ist das Problem des Object-Pickings, d.\,h. die Auswahl des zu manipulierenden Objekts vom Bildschirm. Auch dies wäre mit der oben beschriebenen Methode, die die Gesten der Kameramanipulation verwendet, nur schwierig und umständlich realisierbar gewesen. Das Object-Picking bietet jedoch einen anderen Weg, einen Ansatz für eine Objektmanipulationsgeste zu finden. Ein alltägliches und dem Benutzer bekanntes Beispiel hierfür ist das einfache Greifen nach einem Gegenstand -- dies motiviert auch den Namen \glqq{}GRAB\grqq{}-Geste. Übertragen in eine Geste ließe sich dies durch eine erhobene und geschlossene Hand definieren. In weiterer Analogie zum Alltagsbeispiel sollte ein Hin- und Herbewegen der Hand, mit der gegriffen wurde (der \glqq Kontroll-Hand\grqq{}) auch das Objekt hin"= und herbewegen. Die Rotation führt jedoch bei dieser Geste zu einem Problem: Mit der geschlossenen Hand ist die Erkennung der Orientierung des entsprechenden Gelenks durch die Kinect zu schlecht, um an dieser Stelle sinnvoll Verwendung zu finden. Im Programm äußerte sich dies einerseits durch ein Ausbleiben der Auswirkungen vorgeführter Rotationen, andererseits aber auch durch ein starkes Rauschen. Probleme dieser Art treten auch an anderer Stelle auf und sind behandelbar (dies wird in Abschnitt \ref{sec:robustheit} deutlich), hier jedoch wurden die Ergebnisse so schlecht, dass eine andere Gestendefinition nötig war. Die Ergebnisse wurden direkt deutlich besser, wenn die GRAB"=Geste durch eine gehobene und offene (!) Hand definiert wurde. Die größere Fläche bietet der Kinect mehr Anhaltspunkte und verbessert so die Genauigkeit für die Orientierung, etwa wenn die Handfläche gekippt bzw. gedreht ist.
		\par
		\begin{figure}[H]
		\centering
		\includegraphics[width=.8\textwidth]{pictures/fly_.png}
		\caption{Die Flug-Geste, mit und ohne eingezeichnetes Skelett und HandStates.}\label{fig:flyg}
		\end{figure}
		\item[FLY\_GESTURE] (siehe Abb. \ref{fig:flyg})\par
		Im Rahmen der Tests mit einem Beispielobjekt wurde schnell deutlich, dass es auch eine einfache Möglichkeit geben sollte, Bewegungen über etwas weitere Strecken durch den Raum zu vollführen, ohne dabei ständig zwischen dem Vorführen einer Geste und einem \glqq Nachgreifen\grqq{} wechseln zu müssen. Eine übliche Lösung für eine solche Aufgabe ist ein Flugmodus. Im konkreten Anwendungsfall sollte dann das Vorführen einer besonderen Geste bewirken, dass die Kamera losfährt und erst anhält, wenn die Geste nicht mehr präsentiert wird.\par 
		Die FLY"=Geste entspricht dem Ausstrecken beider Arme vor den Körper, sodass sich die Hände mehr oder weniger am selben Punkt im 3D"=Raum befinden. Passiv findet bei dieser Geste im Programm eine Bewegung nach vorne statt. Durch Schwenken der Arme soll der Nutzer dabei die Richtung der Bewegung beeinflussen können, d.\,h. ein Zeigen der Arme nach oben bewirkt, dass die Bewegung immer weiter nach oben gezogen wird, während mit einem Zeigen nach links oder rechts eine Kurve geflogen werden kann. Dabei bestimmt der Ausschlag der Arme beim Zeigen (verglichen mit der Ausgangsposition, in der beide Arme genau nach vorne gerichtet sind) die Stärke der Richtungsänderung. Um eine sogenannte Fassrolle durchzuführen oder sich \glqq{}in Kurven legen\grqq{} zu können, kann der Nutzer nebenbei seine Schulterpartie in die entsprechende Richtung kippen. Insgesamt ist die Steuerung des Flugmodus in ihrem Funktionsumfang damit ähnlich zu üblichen Steuerungen von Flugzeugen in Actionspielen oder Simulationen: Es findet eine automatische Bewegung nach vorne statt, die in verschiedene Achsen gekippt werden kann.\par
	Diese erneute Prinzipübertragung macht die Geste intuitiver. Mit Hilfe dieser Fluggeste hat der Benutzer eine im Gegensatzu zur Bewegung über Drehen und Schieben einfache Möglichkeit, sich etwa durch ein System von Gängen in einer 3D-Szene zu bewegen und generell Strecken zurückzulegen, statt Objekte zu betrachten. Werden die Strecken zu lang, kann diese Geste jedoch anstrengend für den Benutzer sein. Bei alternativen Anwendungen wäre dies zusätzlich zu bedenken und eine dem abweichenden Zweck besser angepasste Gestendefinition zu verwenden.
		\par		
		\item[UNKNOWN] Dies enthält alles, was als keine der anderen Gesten erkannt wird. Die Kamera und geladene Objekte sollen, solange diese Geste gezeigt wird, stillstehen.\par
		Neben der naheliegenden Motivation, dass der Nutzer die Szene gegebenenfalls auch bei Ruhe betrachten möchte, dient diese \glqq Geste\grqq{} (oder besser \glqq Nicht-Geste\grqq{}) darüber hinaus noch einem anderen Zweck. Sie kann in andere Gesten, wie beispielsweise Translationen, eingebaut werden, um diese aufzubrechen und \glqq nachgreifen\grqq{} zu können. Erst dies gestattet dem Nutzer, während der Bedienung an Ort und Stelle stehen bleiben zu können.
	\end{description}
	Die Verwendung der Gesten hat ergeben, dass es notwendig ist, bei derartig selbst implementierten Gesten auch eigene Robustheitsmechanismen einzubauen, die die Gestenerkennung gegen Schwankungen der Kinecterkennung (etwa des Status einer Hand) abhärten. Für genauere Informationen hierzu sei auf Abschnitt \ref{sec:robustheit} verwiesen.
	\subsection{Zustandsmaschine}
	Das Programm besteht aus zwei Grundmodi der Manipulation: Einerseits der Manipulation der Kamera und andererseits jener des Objekts. Wir können diese beide Modi als zwei Superzustände auffassen, innerhalb derer sich wiederum unterscheidet, auf welche Art und Weise wir manipulieren. Die Zustandsmaschine dient einerseits der Kapselung und Modularisierung der von uns bereitgestellten Funktionen und bildet andererseits die Struktur unserer Manipulationsmodi abstrakt ab.\par 
	Die Zustandsmaschine befindet sich zu jedem Zeitpunkt in einem Zustand. In diesem Zustand findet eine Berechnung der Parameter statt, die unser Programm zurückgibt, die wiederum die Manipulation beschreiben, die ausgeführt werden soll. Dies geschieht durch Auswertung der gesehenen Geste und die Berechnung entscheidender Größen, u.\,U. unter Einbeziehung der Werte vergangener Frames. Schließlich erfolgt basierend auf der präsentierten Geste ein Zustandswechsel am Ende eines Berechnungsschritts. Die Zustandsmaschine ist in Abb. \ref{fig:sm} zu sehen. Im Folgenden erklären wir die Zustände, ihre Semantik und die enthaltenen Berechnungen etwas näher:
	\begin{description}
		\item[IDLE] Dieser Zustand entspricht dem Initialzustand unserer Zustandsmachine. Er ist eine Art Default-Zustand, in dem keine Kamera- und auch keine Objektmanipulation (genauer: keine Berechnung überhaupt) vorgenommen wird. Der Zustand wird betreten, wenn keine der vordefinierten Gesten sicher genug erkannt wurde. Durch Ausführung der entsprechenden Gesten gelangt man zurück in die anderen Zustände.
		\item[CAMERA\_TRANSLATE] Dieser Zustand gehört zur Kameramanipulation. In ihm werden gemäß der oben erklärten Geste die Parameter zur Kamerabewegung bestimmt. Ziel ist die direkte Übertragung der Handbewegungen des Benutzers auf die Kamerabewegung, die im lokalen Raum der Kamera stattfindet. Wir berechnen dazu aus den gepufferten Positionswerten von linker und rechter Hand die diskrete Ableitung, die uns ein Maß für die Geschwindigkeit der Bewegung liefert. Ebenso erhält man daraus die Richtung, in die die Hände bewegt wurden. Die Geschwindigkeit wird dann mit der vergangenen Zeit zwischen dem aktuellen und dem vorhergehenden Frame multipliziert, um wieder einen Entfernungswert zu erhalten. Daraus entstehen die Translationsparameter für die $x$"=, $y$"= und $z$"=Richtungen, die für diesen Zustand unsere \texttt{motionParameters} definieren.
		\item[CAMERA\_ROTATE] Dieser Zustand gehört ebenfalls zur Kameramanipulation. Analog zu oben wird hier die Rotation vorbereitet. Die Rotation ist konzeptionell nahe in der Translation. Während die Geste aktiv ist, wird eine Achse zwischen linker und rechter Hand des Benutzers bestimmt. Bewegt der Benutzer die Hände, so verändert sich die Achse. Ein Quaternion, der die Achsen ineinander überführt, wird berechnet, und analog zur Translation als Geschwindigkeit interpretiert. Dieser proportional zur verstrichenen Zeit zum vorhergegangenen Frame verkleinert und auf die Kamerarotation angewandt.
		\item[OBJECT\_MANIPULATE] Die Objektmanipulation realisiert das einhändige Packen eines virtuellen Objektes mit einer einzigen Hand. Dabei werden Aspekte der Translation und Rotation abgewandelt verwendet. Die Bewegung der packenden Hand wird sehr ähnlich zur Kamera-Translation direkt übertragen. Im Gegensatz zu CAMERA\_TRANSLATE wird diese allerdings auf das Objekt angewandt. Außerdem soll jede Rotation der packenden Hand auf das Objekt übertragen werden. Dazu wird Kinect::JointOrientation genutzt, welches die Orientierung der Hand im Raum als Quaternion bereitstellt. Es wird eine Art Differenz zwischen der Orientierung eines Frames und den gepufferten Vorgängerorientierungen gebildet. Dies findet statt, indem die Vorgängerorientierungen invertiert auf die aktuelle Orientierung angewandt werden. Wie zuvor kann diese Differenz als Geschwindigkeit interpretiert werden. Bei der Anwendung auf das Objekt muss jedoch beachtet werden, dass eine Drehung der Hand beispielsweise um ihre Z-Achse nicht unweigerlich auch einer Drehung des Objektes um dessen Z-Achse haben soll. Für natürliches Verhalten muss die Rotation selbst noch abhängig von Objekt- sowie Kamerarotation gedreht werden.
		\item[FLY] Dieser Zustand wurde nachträglich eingeführt, als die Notwendigkeit eines Flug-Modus deutlich wurde. Er wird mittels Vorführung der FLY"=Geste betreten und analog zu den anderen Zuständen verlassen. Der Flugmodus soll eine durchgehende Bewegung realisieren, ohne dass der Benutzer sich permanent bewegt. Sie bildet primär eine Alternative zum wiederholten Anwenden von CAMERA\_TRANSLATE. Während die Geste aktiv ist, wird eine Translation mit festen Werten nach vorn vorgenommen. Außerdem ist es möglich, durch die Neigung der nach vorn gestreckten Arme zu lenken. Dazu wird die mittlere Handposition mit einem Referenzpunkt auf dem Körper des Benutzers verglichen. Die Abweichung der Position von einer neutralen Ausgangsposition bestimmt die Lenkrichtung. Die Stärke der Auslenkung bestimmt dabei die Stärke der Rotation. Um schnellere Drehungen zu gewährleisten, werden starke Auslenkungen noch zusätzlich verstärkt. Analog zur CAMERA\_ROTATE wird ein Quaternion berechnet, welcher die neutrale Position auf die gegebene Position abbildet. Er wird wie üblich angewandt. Außerdem wird die Neigung der Achse von linker Schulter zu rechter Schulter bestimmt. Diese wird in eine kontinuierliche Seitwärtsrolle übersetzt.
	\end{description}
	Zur Verdeutlichung sei darauf hingewiesen, dass von jedem Zustand zu jedem anderen übergegangen werden kann, wobei dieser Übergang lediglich anhand erkannter Gesten erfolgt: Wird eine unserer Gesten erkannt (Details siehe Abschnitt \ref{sec:robustheit}), so wird der zugehörige Zustand betreten. Da unser Programm darauf ausgelegt ist, während des Event-Loops einer Hauptanwendung zu laufen, besteht die Zustandsmaschine ab ihres Starts permanent (bzw. bis zum Ende der Hauptanwendung) und besitzt keinen Finalzustand.\par
	Genaueres zum Aussehen der State-Machine als Datenstruktur ist in Abschnitt \ref{sec:ds} zu finden.
	\begin{figure}[h!]
	\centering
	\usetikzlibrary{positioning}
\resizebox{\textwidth}{!}{
\begin{tikzpicture}

%%IDLE-State
\umlbasicstate[x=8,y=4, name=IDLE, fill=white]{IDLE}

%%INIT-State
\umlstateinitial[below=2cm of IDLE.south, name=INIT]
	\umltrans{INIT}{IDLE}

%%Superstate Kamera-Manipulation
\begin{umlstate}[x=0,y=8,name=CAM, fill=black!20]{Kamera-Manipulation}
	%%%%%%%%%%%%%%
	%% Zustände %%
	%%%%%%%%%%%%%%
	%%Zustand Kamerabewegung
	\umlbasicstate[x=0,y=0, name=CAMTRANS, fill=white]{CAMERA\_TRANSLATE}
	%%Zustand Kameradrehung
	\umlbasicstate[x=0,y=-4, name=CAMROT, fill=white]{CAMERA\_ROTATE}
	%%Zustand FLY
	\umlbasicstate[x=0,y=-8, name=FLY, fill=white]{FLY}
	
	%%%%%%%%%%%%%%%%%%
	%% Transitionen %%
	%%%%%%%%%%%%%%%%%%
	%\umlHVHtrans[anchor1=20,anchor2=170,arg={ROT...},pos=1.5]{CAMTRANS}{CAMROT}
	%\umlHVHtrans[anchor1=-170,anchor2=-20,arg={TRA...},pos=1.5]{CAMROT}{CAMTRANS}
	
%	\umltrans[recursive=-120|-170|3cm, recursive direction=bottom to left, arg={TRANSLATE\_GESTURE},pos=1.3]{CAMTRANS}{CAMTRANS}
	%\umltrans[recursive=-10|-60|3cm, recursive direction=right to bottom, arg={ROTATE\_GESTURE},pos=2.6]{CAMROT}{CAMROT}
\end{umlstate}


%%Superstate Objekt-Manipulation
\begin{umlstate}[x=8,y=8,name=OBJ, fill=black!20]{Objekt-Manipulation}
	%%%%%%%%%%%%%%
	%% Zustände %%
	%%%%%%%%%%%%%%
	%%Zustand Kamerabewegung
	\umlbasicstate[y=0,name=OBJMAN, fill=white]{OBJECT\_MANIPULATE}
	%%%%%%%%%%%%%%%%%%
	%% Transitionen %%
	%%%%%%%%%%%%%%%%%%
	%\umltrans[recursive=-40|-140|2cm, recursive direction=bottom to bottom, arg={GRAB\_GESTURE},pos=1.5]{OBJMAN}{OBJMAN}
\end{umlstate}

\umltrans[recursive=-10|-60|3cm, recursive direction=right to bottom]{OBJMAN}{OBJMAN}
\umltrans{CAMROT}{OBJMAN}
\umltrans{FLY}{OBJMAN}
\umltrans[anchor1=110,anchor2=-110]{IDLE}{OBJMAN}
\umltrans{CAMTRANS}{OBJMAN}

\node at (4.6,8.6) {G};
\node at (5.2,6.6) {G};
\node at (5.8,6.5) {G};
\node at (7.4,6.5) {G};
\node at (11.7,7.6) {G};

\umltrans[recursive=-10|-60|3cm, recursive direction=right to bottom]{IDLE}{IDLE}
\umltrans{CAMROT}{IDLE}
\umltrans[anchor1=15,anchor2=-140]{FLY}{IDLE}
\umltrans{OBJMAN}{IDLE}
\umltrans{CAMTRANS}{IDLE}

\node at (8.3,5.6) {U};
\node at (9.7,4.4) {U};
\node at (6.4,4.8) {U};
\node at (6.1,4.6) {U};
\node at (7,3) {U};

\umltrans[recursive=-60|-120|1cm, recursive direction=bottom to bottom]{FLY}{FLY}
\umltrans{CAMROT}{FLY}
\umltrans{IDLE}{FLY}
\umltrans[anchor1=-120,anchor2=35]{OBJMAN}{FLY}
\umlHVHtrans[anchor1=-190,anchor2=-160,arm1=-2cm]{CAMTRANS}{FLY}

\node at (-1,-1.5) {F};
\node at (-1.5,-.5) {F};
\node at (0.4,1.7) {F};
\node at (2.5,2) {F};
\node at (2.2,1.6) {F};

\umltrans[recursive=-170|-190|.5cm, recursive direction=left to left]{CAMROT}{CAMROT}
\umltrans[anchor1=-110,anchor2=110]{CAMTRANS}{CAMROT}
\umltrans[anchor1=-163,anchor2=-10]{IDLE}{CAMROT}
\umltrans[anchor1=-160,anchor2=35]{OBJMAN}{CAMROT}
\umltrans[anchor1=110,anchor2=-110]{FLY}{CAMROT}

\node at (-.8,5.8) {R};
\node at (-.8,2.8) {R};
\node at (-2.6,5) {R};
\node at (1.3,5.8) {R};
\node at (2.6,3.7) {R};

\umltrans[recursive=-10|5|1cm, recursive direction=right to right]{CAMTRANS}{CAMTRANS}
\umltrans{CAMROT}{CAMTRANS}
\umltrans[anchor1=140,anchor2=-20]{IDLE}{CAMTRANS}
\umltrans[anchor1=-175,anchor2=-5]{OBJMAN}{CAMTRANS}
\umlHVHtrans[anchor1=-190,anchor2=-170,arm1=-3cm]{FLY}{CAMTRANS}

\node at (3.6,8.8) {T};
\node at (3.4,7.3) {T};
\node at (-3.4,7.4) {T};
\node at (.5,6.8) {T};
\node at (4,7.8) {T};

%\umlVHVtrans[arm1=-1cm,anchor1=-60,anchor2=-150,arg={GRAB\_GESTURE},pos=1.4]{IDLE}{OBJMAN}
%\umlVHVtrans[arm1=-2cm,anchor1=-30,anchor2=-120,arg={UNKNOWN\_GESTURE},pos=1.5]{OBJMAN}{IDLE}

%\umlVHVtrans[arm1=4cm,anchor1=130,anchor2=-45,arg={TRANSLATE\_GESTURE},pos=0.5,name=IDLETOCAM]{IDLE}{CAMTRANS}
%\umlVHVtrans[arm2=3.75cm,anchor2=120,anchor1=-40,arg={UNKNOWN\_GESTURE},pos=2.1,name=CAMTOIDLE]{CAMTRANS}{IDLE}
%\umlpoint{CAMTOIDLE-2}
%\umlVHtrans[anchor1=-150]{CAMROT}{CAMTOIDLE-2}
%\umlpoint{IDLETOCAM-1}
%\umlHVHtrans[arm1=-2cm,anchor1=170]{OBJMAN}{IDLETOCAM-1}

%\umlVHVtrans[arm1=2.5cm,anchor1=55,anchor2=-145,arg={ROTATE\_GESTURE},pos=0.5,name=IDLETOCAM2]{IDLE}{CAMROT}
%\umlpoint{IDLETOCAM2-1}
%\umlHVHtrans[arm1=-3cm,anchor1=-170]{OBJMAN}{IDLETOCAM2-1}

%\umlVHVtrans[anchor1=-140,anchor2=155,arm1=-4cm,arg={GRAB\_GESTURE},pos=0.999,name=CAMTOOBJ]{CAMROT}{OBJMAN}
%\umlpoint{CAMTOOBJ-4}
%\umlVHVtrans[anchor1=-35,arm1=-1cm]{CAMTRANS}{CAMTOOBJ-4}
%\umltrans[recursive=90|180|5cm,recursive direction=top to left,arg={UNKNOWN\_GESTURE},pos=.5]{IDLE}{OBJMAN}
%\umlVHtrans{IDLE}{OBJMAN}
\end{tikzpicture}
}

	\caption{Die Zustandsmaschine. Der nachträglich eingefügte Zustand FLY ist mit allen anderen Zuständen über die entsprechende Geste verbunden und wird von allen Zuständen durch Präsentieren der FLY-Geste erreicht. Aus Gründen der Übersichtlichkeit wurde auf das Einzeichnen dieser Kanten verzichtet.}\label{fig:sm}
	\end{figure}

		\subsection{Robustheit und Pufferung}\label{sec:robustheit}
	Wie wir vorangegangen festgestellt haben, sind einige der Mechanismen, die wir implementieren wollen anfällig gegenüber qualitativ niedrigwertigen Kinectdaten. Tests mit der Kinect haben folgende kritische Situtationen ergeben:
	\begin{itemize}
		\item Gelenke und Skelettbestandteile in der Nähe von Objekten und anderen Personen. Diese können falsch oder verzerrt erkannt werden. So kann etwa die erkannte Handposition zwischen zwei Kinectframes Raumunterschiede von mehreren Metern aufweisen und zurückspringen.
		\item Status der Hände. Auch bei durchgängiger Aufrechterhaltung eines Handzustands kann es passieren, dass die Kinect vereinzelt falsche Zuweisungen trifft oder keine Zuweisung möglich ist.
	\end{itemize}
	Beide Situationen lassen sich behandeln, indem Entscheidungen unseres Programms, nicht nur vom augenblicklichen Rückgabewert der Kinect abhängen, sondern auch einige vergangene Werte mit einbeziehen. So kann ermittelt werden, ob der aktuelle Wert (mit hoher Wahrscheinlichkeit) ein zu ignorierender Ausreißer ist. Dazu wird ein Ringpuffer verwendet und an den entsprechenden Stellen im Quellcode ein gewichtetes Mittel über den Pufferinhalt gebildet, wobei neuere Einträge mit deutlich größerem Gewicht eingehen. Für Gesten kann dann mit einer bestimmten Zuverlässigkeit eine Zuordnung getroffen werden, für Raumpositionen stellt dieses Mittel eine Glättung dar. Dies hat den positiven Nebeneffekt, dass die endgültige Anwendung der errechneten Parameter auf die Kamera bzw. das Objekt ebenfalls geschmeidiger werden.
%
%
\clearpage
\section{Bemerkungen zum Quellcode}\label{sec:code}
In diesem Abschnittwerden wesentliche Stellen bzw. Strukturen des Programmcodes erläutert. Dazu wird auf die verwendeten Datenstrukturen, Variablen und Funktionen eingegangen und ihr Zusammenspiel grob illustriert. Schließlich wird in diesem Abschnitt auch auf das Einbinden des Programmes zur Verwendung in fremder Software eingegangen.
		\subsection{Wichtige Datenstrukturen, Variablen und Funktionen}\label{sec:ds}
	\subsection{Ergänzungen zum Zusammenspiel und Ablaufskizze}
In diesem Abschnitt wird der Ablauf des Programms skizziert. Dabei wird davon ausgegangen, dass KinectControl bereits instanziiert und durch Aufruf der \texttt{init()}-Funktion initialisiert wurde. Weiterhin gilt die Vorstellung, dass die \texttt{run()}-Methode im Main-Loop der umgebenden Software ausgeführt wird. Genaueres zur Einbindung ist Abschnitt \ref{sec:einbinden} zu entnehmen.\par 
Sollte kein Master bestimmt sein, so werden für den Master zunächst eindeutige Defaultwerte für ID (der negative Wert $-1$) und $z$"=Entfernung zur Kamera (der maximale Wert des Floatdatentyps) angenommen.\par 
Dann wird versucht, über den \texttt{BodyFrameReader} der Kinect einen aktuellen Kinect-Frame abzugreifen. Sollte dies fehlschlagen, so werden die alten Bewegungsparameter einfach weiterverwendet. Im Falle des Erfolgs versuchen wir auf die Körpertracking-Daten der Kinect zuzugreifen, was im Fehlerfall analog behandelt wird. Gerade die erste Art von Fehlschlag (der Versuch, einen Kinect-Frame zu holen, obwohl kein solcher vorhanden ist) tritt dabei tatsächlich häufig ein, da die Kinect nur etwa 30 Frames pro Sekunde liefern, wohingegen der Tick des Main Loops üblicherweise deutlich darüber liegt.\par 
Nachfolgend findet im Code die Masterbehandlung statt. Hier werden vier Fälle unterschieden, die gesondert erklärt werden sollen:
\begin{itemize}
%
\item Der weitere Programmablauf sei nun zunächst für den Fall geschildert, dass kein Master eingespeichert wird oder wurde. In diesem Falle ist der nächste relevante Schritt das Iterieren über die Körper, die die Kinect zurückgeliefert hat. Dabei holen wir uns die genauen Körperdaten, namentlich die Positionen und Orientierungen der Gelenke. In der genannten Situation (vor jeglicher Masterfestlegung) wird als Basislösung der primitive $z$-Test zur Festlegung einer Person, die das Programm steuert, verwendet. Dazu wird das Master-Objekt einfach mit den Werten des Körpers belegt, dessen Kopf den niedrigsten $z$"=Wert aufweist. 
\item Das Programmverhalten ändert sich, wenn durch Betätigen der Einspeichertaste X ein Master festgelegt werden soll. Eine entsprechende Abfrage im Main-Loop des unterliegenden Programmes ruft die \texttt{assignMaster()}-Funktion von KinectControl auf, die die für die Masterfestlegung wichtigen Parameter initialisiert. Vor allem werden die booleschen Variablen \texttt{masterDetermined} und \texttt{collectFrames} auf \texttt{true} gesetzt und je nach Länge des Tastendrucks eine Variable hochgezählt, die die Anzahl der Samples für die Vermessung bestimmt. Fall zwei bei der Masterbehandlung entspricht dann der Belegung der beiden Variablen \texttt{masterDetermined} und \texttt{collectFrames} mit \texttt{true}. In diesem Falle wird beim Iterieren über die Körper zunächst ein Skelett in Standardpose gesucht. Die ID dieses Skeletts wird gespeichert und die zugehörige Person soll als Master eingespeichert werden. Befinden sich mehrere Skelette in Standardpose, so wird das indexmäßig erste im \texttt{trackedBodies}-Array für die Masterfestlegung verwendet. Nachdem die genannte ID und damit der künftige Master festgelegt ist, werden solange Körpermerkmale gesammelt, wie durch den Framezähler aus \texttt{assignMaster()} vorgegeben. Dazu wird die \texttt{collectBodyProperties()}-Funktion aus der \texttt{Person}-Klasse aufgerufen, die wie weiter oben ausgeführt ein festes Set an Körpermerkmalen puffert. Sollte der designierte Master dabei die Standardpose verlassen, wird die bisherige Sammlung mittels \texttt{deleteCollectedBodyProperties()} vollständig verworfen und die ID-Festlegung durch Standardposensuche vom Anfang erneut durchgeführt. War der Frame hingegen gut, wird die Anzahl noch zu sammelnder Frames dekrementiert. Sind genau so viele Frames gesammelt, wie durch \texttt{assignMaster()} vorgegeben wurde, so wird die Masterfestlegung durch Rücksetzen von \texttt{collectFrames} auf \texttt{false} und Aufruf von \texttt{calculateBodyProperties()} (ebenfalls aus der \texttt{Person}-Klasse) beendet. Konzeptuell erstellt letztere Funktion aus den vorher gesammelten Daten einen charakteristischen Vektor für die eingespeicherte Person, den sie im Master-Objekt ablegt.
\item Der dritte Fall in Sachen Masterfestlegung schließt sich sodann an, wenn also ein Master bestimmt wurde (\texttt{masterDetermined} ist \texttt{true}), jedoch keine Samples mehr gesammelt werden müssen (\texttt{collectFrames} ist \texttt{false}), weil der charakteristischen Vektor bereits gebildet wurde und verwendet werden kann. Dann muss hinsichtlich des Masters nur noch etwas getan werden, falls der Master zwischenzeitlich aus dem Tracking verschwunden ist, was vor der Iteration über die Körper überprüft wird. In diesem Falle wird \texttt{searchForMaster} gesetzt. Für die 0-1-Flanke von \texttt{searchForMaster} wird ein spezielles \texttt{lostMaster}-Flag gesetzt. In der Iteration über die Körper wird dann nach Körpern in Standardpose Ausschau gehalten und für diese Körper eine Sammlung von Abweichungswerten zum charakteristischen Vektor des Masters aufgebaut. Auf Codeebene wird dies durch ein Array von Puffern, \texttt{deviationBuffer[]}, gelöst. Die Abweichungen werden durch Aufruf der \texttt{compareBodyProperties()}-Methode der \texttt{Person}-Klasse berechnet. Das korrekte Weitersammeln in einem angefangenen Puffer kann über die von der Kinect vergebene, eindeutige \texttt{trackingId} gesichert werden. Die Puffer haben eine feste Größe und werden, wenn sie voll -- aktuell entspricht dies 20 Samples -- sind, durch Aufruf von \texttt{evaluateDeviationBuffer()} gemittelt. Ergebnis ist ein gemittelter Abweichungswert vom charakteristischen Vektor des Masters. Unterschreitet dieser eine empirisch festgelegte Grenze, nehmen wir an, das die präsentierte Person der Master ist und setzen das \texttt{master}-Objekt unserer Zustandsmaschine entsprechend.\par 
\item Als letzter Punkt in Sachen Masterbehandlung ist noch das Verhalten im Falle eines Masterverlusts (\texttt{lostMaster} ist dann \texttt{true}) zu klären. In diesem Falle werden sämtliche Bewegungsparameter und der Zustand der State-Machine zurückgesetzt. Außerdem wird \texttt{lostMaster} wieder auf \texttt{false} gesetzt, was den Effekt hat, dass die Variable tatsächlich genau die 0-1-Flanke von \texttt{searchForMaster} einfängt.\par
\end{itemize}
Im Programmtext und Ablauf folgt nach der Masterbehandlungsphase nun die Masterauslesung zur Steuerung des Programms. Sie findet statt, falls ein aktiver Master vorhanden ist. Von diesem werden sodann die Gelenke mit ihren Orientierungen und die Status der Hände ausgelesen. Die entsprechenden und wichtigen Merkmale werden im \texttt{master}-Objekt (z.\,T. nach Plausibilitätsüberprüfungen und gepuffert) abgelegt. Dann findet die Kernberechnung der Zustandsmaschine statt (siehe Abb. \ref{fig:ber}). In \texttt{bufferGestureConfidence()} wird aus den Joints und HandStates unter Verwendung eines Puffers ein Konfidenzvektor für die Gesten erstellt. In \texttt{compute()} findet die Berechnung der \texttt{motionParameters} entsprechend des aktuellen Zustands und der Gelenkdaten statt. Die Funktion \texttt{switchState()} nimmt schließlich anhand des Gestenkonfidenzvektors gegebenenfalls einen Zustandsübergang vor. Konzeptuelle Erläuterungen zur Funktionsweise dieser Funktionen sind in den vorangegangenen Abschnitten zu finden.\par\medskip
\begin{figure}[h]
\centering
\includegraphics[width=.8\textwidth]{pictures/statemachine-ber.png}
\caption{Der Code, der den Berechnungsschritt der State-Machine darstellt.}\label{fig:ber}
\end{figure}
Am Ende der \texttt{run()}-Methode wird der Frame freigegeben und die bei \texttt{compute()} berechneten \texttt{motionParameters} zurückgegeben. Wie genau sie schließlich verwendet werden, entscheidet das unterliegende Programm.

	\subsection{Einbinden}\label{sec:einbinden}
Das vorliegende Projekt kompiliert eine statische Bibliothek (.lib). Diese ist (in Visual Studio) als \glqq{}Additional Dependency\grqq{} für das Linking einzutragen.\par\medskip
Das Programm, das die Library verwendet, sollte mittels \[\texttt{KinectControl kinectControl;}\]ein KinectControl-Objekt anlegen und in seiner Initialisierung durch Aufruf seiner \texttt{init()}-Funktion mitinitialisieren.\par 
Im Main-Loop oder Event-Loop des Programms werden per \[\texttt{MotionParameters motionParameters = kinectControl.run();}\] die Parameter (Translation, Rotation und Ziel) geholt und können ausgewertet und entsprechen angewendet werden.\par 
Weiterhin sollte \texttt{kinectControl.assignMaster()} irgendwie angestoßen werden können, z.\,B. über ein bestimmtes Tastendruck-Event.
%
%
\clearpage
\section{Schlussbemerkungen}\label{sec:schluss}
In diesem Abschnitt soll diskutiert werden, in welchem Umfang die entstandene Anwendung die an sie gestellten Anforderungen erfüllt und andererseits bemerken, wie die Kinect selbst überhaupt für eine solche Aufgabe geeignet ist.
%
%
\subsection{Eignung der Software}\label{sec:programm}
Die erstellte Software erfüllt die Aufgabenstellung in einem zufriedenstellenden Maße: Es ist möglich, die 3D-Szene und enthaltene Objekte per Geste zu Steuerung und diese Steuerung technisch einfach durchzuführen. Nach Rückkopplung aus diversen Selbstversuchen hat sich gezeigt, dass das Steuerempfinden für den Nutzer in der subjektiven Wahrnehmung in einem guten Maße präzise ist, was die Übertragung der Handbewegungen auf Objekte oder die Kamera betrifft. Bei schlechten Daten (siehe Abschitt \ref{sec:rob1}) kann es dennoch passieren, dass die Steuerung zittert oder springt. Um dies zu minimieren sollte zusätzlich auf Kooperation des Nutzers gesetzt werden in dem Sinne, dass er z.\,B. nicht zu weite Kleidung wie offene Jacken trägt und die Hände bei der Steuerung vorzugsweise neben (und nicht vor) dem Körper hat. Dies kommt dann der Kinect bei der Erkennung entgegen. Bei der einhändigen Objektrotation ist der Einfluss der Kinectgüte am stärksten sprbar, vor allem beim Kippen nach vorne oder hinten.\par
Die Mastererkennung funktioniert ebenfalls wie vorgesehen, es kann jedoch im ungünstigsten Falle passieren, dass für den Master ein fehlerhaftes Skelett mit falschen Proportionen beobachtet wird. Dann wird er gegebenenfalls nicht wiedererkannt und neu eingespeichert werden muss.
%
\subsection{Eignung der Kinect}
Wie sich durch das Projekt herausgestellt hat, ist die Kinect mit einigen Abstrichen bei der Datenqualität für $1$"=zu"=$1$"=Abbildungen geeignet. Demgegenüber steht aber ihre Verfügbarkeit, der einfache Zugang und der Preis (vgl. wieder \ref{sec:kinect}).\par 
Rückblickend wäre eine Steuerung durch diskrete Gestenerkennung ohne $1$"=zu"=$1$"=Zusammenhang einfacher zu implementieren gewesen, bspw. so, dass eine bestimmte Geste eine Bewegung vordefinierter Distanz in eine der Achsenrichtungen bewirkt. Dies würde zwar viele der Probleme aus Abschnitt \ref{sec:robustheit} vermeiden, trotzdem bleibt eine Direktabbildung wegen der größeren Freiheiten angenehmer. Außerdem gibt es, während die Bewegung einfach zu berwerkstelligen wäre, keinen natürlichen Ansatz einer solchen Gestenfindung für eine Rotation im Raum.\par 
Trotz starker Glättung ist es nicht gelungen, den Jitter unter Erhalt der Echtzeitabbildung vollständig zu eliminieren, sodass in Resten noch für den Nutzer sichtbar ist, wenn auch nur in einem subtilen Maße.\par 
Bei der Mastererkennung ist die größte Problematik die im vorangegangenen Abschnitt \ref{sec:programm} genannte. Dazu kommt, dass es mit den hier entwickelten Methode schwierig ist, sehr ähnliche Personen zu differenzieren -- wobei bessere Systeme z.\,B. bei der Präsentation eineiiger Zwillinge vermutlich auch an ihre Grenzen kommen. Hierzu bestand jedoch kein Zugriff auf passende Testpersonen. Mit größerem Aufwand lassen sich jedoch bessere Erkennungsquoten erzielen, vergleiche dazu etwa \cite{appearance}, \cite{biomid} und \cite{bodyprop}. Es gibt auch gänzlich andere Ansätze, wie etwa über den Gang (siehe \cite{gait}) oder das Gesicht (siehe \cite{face}). In einem gewissen Rahmen findet dabei stets ein Abtausch zwischen Sicherheit der Erkennung und Laufzeit statt, der je nach Anwendung zu berücksichtigen ist.\par 
An dieser Stelle sei damit geschlossen, dass die Kinect für den gegebenen Zweck verwendet werden kann, aber einiges an Nachbehandlung der Daten erfordert. Zu sicherheitsrelevanten Zwecken sollte der Einsatz höherwertiger Systeme in Betracht gezogen werden.
%
\subsection{Verbesserungen}
Die entstandene Software wird den Anforderungen aus Abschnitt \ref{sec:aufg} und diversen Erweiterungen gerecht. Dennoch soll hier Ausblick auf potenzielle Verbesserungen gegeben werden.
Zunächst wäre es in der Zukunft natürlich wünschenswert, wenn sämtliche Stubfunktiononalität und der geplante Funktionsumfang implementiert würden, sobald das einbindende Grundprogramm sie unterstützt.\par 
Die Parameter des Programms ließen sich in der Zukunft weiter optimieren. Eventuell sind dafür auch Tests in einem größerem Rahmen -- z.\,B. mit etwa 50 Testpersonen -- nützlich. Diese würden den zusätzlichen Zweck erfüllen, einen besseren Einblick in die Unterschiede zwischen verschiedenen Skeletten zu gewinnen und so bspw. neue Körpermerkmale für die Erkennung zu gewinnen oder weniger aussagekräftige auszusondern. Darüber hinaus bieten derartige Tests noch mehr individuelle Erfahrungsberichte, die genutzt werden können, um die Intuitivität und das Eintauchen des Nutzers in die virtuelle Umgebung zu verbessern.\par 
In der Hinsicht auf Programmparameter wären für den Anwender auch zusätzliche Schnittstellen nach außen wünschenswert, z.\,B. um die Genauigkeit der Mastererkennung von außen vorgeben zu können. Während der Arbeit war hier etwa im Gespräch, die Einspeicherung so lange fortzusetzen, bis die Standardabweichung aller gesammelten Daten unter eine vorgegebene Konvergenzschranke gerät und die erlaubte Abweichung damit anstelle der empirischen Feststellung präzise vorgegeben werden kann. Dies ist jedoch ohne Feedback für den Nutzer nicht sinnvoll möglich, da dieser sonst über keinerlei Wissen verfügt, wie lange die Masterfestlegung (noch) dauern wird. Daher sollte im Vorhinein ein komplett funktionstüchtiges Eventsystem und Möglichkeiten der Ausgabe in der Grundanwendung bereitstehen. Weiterhin löst dies nicht das Problem schlechter Skelettzuweisungen und es kann weiterhin passieren, dass Skelette mit falschen Proportionen eingespeichert werden. Zusammenfassend war hier das erwartete Aufwand-Nutzen-Verhältnis zu niedrig und es wurde nicht zuletzt aus Zeitgründen auf eine Implementierung verzichtet.
Darüber hinaus gibt es Erweiterungen der Funktionalität, die die Bedienerfahrung verbessern würden. Dies betrifft etwa ein Zurücksetzen von Kamera und Objekten, ggf. per Geste; die Skalierung und das Löschen von Objekten mittels Gesten und sicher viele weitere. Einige solcher Wunschfunktionalitäten ergeben sich erst im Laufe der tatsächlichen Verwendung der Software, wie es etwa bei der hinzugekommenen Aufgabe, einen Flugmodus zu implementieren der Fall war.
%
\subsection{Fazit}
Trotz der Probleme der Kinect stellt das hier entwickelte Programm mit vertretbarem Aufwand eine verwertbare Softwarelösung der Aufgabenstellung und insbesondere für Präsentationen wie an einem Tag der offenen Tür einen Mehrwert dar.\par 
Sofern hohe Genauigkeit notwendig ist oder die Anwendung irgendwelchen Sicherheitskriterien genügen muss, sei jedoch die Nutzung eines höherwertigen Trackingsystems zuungunsten der Kinect empfohlen.\par 
%TODO verwendete Werkzeuge

\newpage
\nocite{*}
\printbibliography
\end{document}