\subsection{Aufgabenstellung}\label{sec:aufg}
Aus der in Abschnitt \ref{sec:grundl} genannten Motivation heraus, ist es Ziel des Projektes, die Steuerung der gegebenen 3D-Anwendung hinsichtlich ihrer Präsentation vor einer Zuschauergruppe zu erleichtern und intuitiv zu gestalten. Dabei wird besonderer Wert auf das Eintauchen des Vorführenden in die 3D-Szene gelegt. Diese Anforderung soll durch die Implementierung einer Gestensteuerung erfüllt werden. Nebenbei findet Arbeit mit den noch nicht eingesetzten Tracking-Systemen des Fachgebiets statt, die zukünftige Themen für studentische Arbeiten oder Forschung am Fachgebiet stützen kann. Die vorhandenen Trackingsysteme sind
\begin{itemize}
\item ein professionelles Motion-Capture-Trackingsystem zum Tracken von Raumpunkten über angebrachte Marker und Wands und
\item die Microsoft Kinect 2 zur optischen Gestenerkennung, i.\,W. mittels Kameraaufnahmen einer Infrarotprojektion.
\end{itemize}
Das so ausgewählte Trackingsystem ist also zur Implementierung einer Gestensteuerung der gegebenen Anwendung zu verwenden. Die dafür entwickelte Software soll folgendes leisten:
\begin{itemize}
\item Es soll in der Lage sein, sämtliche Steuerung und Manipulation, die oben beschrieben wurde, durchzuführen, d.\,h. Kamera und Objekte verschieben und drehen können.
\item Die Bedienung soll intuitiv und einfach sein, d.\,h. etwaige Gesten müssen bezüglich der ihnen zugeordneten Aktion einleuchtend und leicht ausführbar sein.
\item Die Steuerung soll ihrem Zweck angemessen genau sein, bestenfalls als glatte 1-zu-1-Übertragung von Handbewegungen auf die Szene.
\item Das entstehende Programm soll möglichst einfach eingebunden und wiederverwendet werden können.
\item Darüber hinaus soll eine Mastererkennung bzw. -verwaltung implementiert sein, d.\,h. ein Verfahren, das garantiert, dass auch nur die dafür vorgesehene Person das Programm steuert und niemand sonst. Hiermit wird ausgeschlossen, dass ein Fremder die Kontrolle über das Programm gewinnen kann und die Vorführung damit -- gewollt oder unbewusst -- behindert. Die als Vorführender ausgezeichnete Person soll auch später wieder erkannt werden können und die Kontrolle wiedererlangen, etwa nachdem sie einen Augenblick lang nicht im von der Kamera abgedeckten Bereich war.
\end{itemize}
Damit ist die vollständige Liste der Anforderungen gegeben und wird hier zur Übersicht nochmals im Kern zusammengefasst:\par\medskip
Ziel ist die Entwicklung einer Software
\begin{itemize}
\item unter Verwendung vorhandener Technik (einem Trackingsystem),
\item die eine Gestensteuerung der gegebenen Anwendung ermöglicht und 
\item dabei nur dem Präsentierenden als ausgezeichneter Person die Steuerung erlaubt.
\end{itemize}
Zur Umsetzung dessen fiel die Wahl auf die Microsoft Kinect 2 als Trackingsystem. Dies hatte vielerlei Gründe:
\begin{itemize}
\item Wegen ihrer kommerziellen Herkunft aus der Spieleindustrie (vgl. Abschnitt \ref{sec:kinect}) kann davon ausgegangen werden, dass sie bereits einem breiteren Publikum bekannt ist und gegebenenfalls auch besonders reizvoll erscheint.
\item Vom Verfahren der 3D-Daten-Gewinnung her (Genaueres in Abschnitt \ref{sec:kinect}) kann die Kinect vom Nutzer direkt und ohne weiterführende Vorbereitungsmaßnahmen verwendet werden. Markerbasierte Trackingsysteme haben diesen \glqq{}Plug-and-Play\grqq{}-Vorteil nicht.
\item Ähnlich zum ersten Punkt liegt hinsichtlich der Kinect aufgrund ihres Bekanntheitsgrades eine sehr gute Quellenlage im Internet vor. Sie verfügt über eine offizielle (wenn auch in Teilen knappe) Online-Dokumentation und da sie mit SDK und API veröffentlicht wurde, finden sich auch in einem breiteren Rahmen zahlreiche Lösungsdiskussionen und -präsentationen von Nutzern im Netz.
\end{itemize}
