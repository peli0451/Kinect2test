\subsection{Grundlagen und Motivation}
	Gegeben ist eine bereits vorhandene 3D-Anwendung, die vom Betreuer des Projektseminars, Herrn M.\,sc. Julian Meder, erstellt wurde. Das Programm wird zu Demonstrationszwecken z.\,B. am Tag der offenen Tür in den Räumlichkeiten des Fachgebiets genutzt. Innerhalb der Anwendung ist es möglich,
	\begin{itemize}
		\item Objekte einzuladen und anzeigen zu lassen sowie
		\item die Kamera (bzw. Kameras) zu manipulieren, d.\,h. zu bewegen, zu rotieren und zu zoomen.
	\end{itemize}
	In Erweiterung soll das Programm auch weiteren Aufgaben der Objektmanipulation dienlich sein, dies beinhaltet insbesondere das Laden mehrerer Objekte in eine Szene und die Manipulation dieser im Sinne von Translation, Rotation, Skalierung und Löschung.\par
	Den oben genannten Demonstrationszweck nimmt das Programm über die Ankopplung des Rechners an einen 3D-Kameraaufbau wahr. Das Programm rendert zwei Ausgabefenster aus zwei verschiedenen, im Sinne der Stereoskopie angeordneten Kamerasichten. Dies wird von den beiden Projektoren im Nebenraum des Präsentationsraums genutzt um ein linkes und ein rechtes Bild von hinten auf einen Schirm zu strahlen, der in die Trennwand der beiden Räume eingelassen ist. Mittels handelsüblicher aus 3D-Kinos bekannten Shutterbrillen können Benutzer und Zuschauer sodann den 3D-Eindruck wahrnehmen.\par
	Die Steuerung des genannten Grundprogramms erfolgt durch den Vortragenden (im Weiteren zumeist Master genannt) über Tastatur und Maus bzw. einen Präsentationspointer. Die Unhandlichkeit sowie fehlende Immersion und Attraktivität dieser Steuerung ist eine Kernmotivation für unser Projekt. Darüber hinaus motivierend ist auch die bloße Auseinandersetzung mit der am Fachgebiet vorhandenen Technik sowie die Untersuchung der Möglichkeiten, diese zielführend einzusetzen.