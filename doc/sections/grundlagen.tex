\subsection{Grundlagen \& Technik}
	Gegeben ist eine bereits vorhandene 3D-Anwendung, die zu Demonstrationszwecken genutzt wird. Innerhalb der Anwendung ist es möglich,
	\begin{itemize}
		\item Objekte zu laden und damit anzeigen zu lassen sowie
		\item die Kamera (bzw. Kameras) zu manipulieren, d.\,h. zu bewegen, zu rotieren und zu zoomen,
	\end{itemize}
zusätzlich geplant ist später
	\begin{itemize}
		\item geladene Objekte manipulieren, in diesem Falle skalieren oder löschen zu können.
	\end{itemize}
	Das Programm rendert dabei zwei Ausgabefenster, in denen die Szene dargestellt ist, wobei die Kameras einen 3D-Aufbau bilden.\par
	Die so beschriebene Ausgabe wird über zwei Projektoren von hinten auf eine Projektionsfläche geworfen -- ein Projektor für die linke Kamera und einer für die rechte. Wird die \glqq Leinwand\grqq{} von vorne durch eine Shutterbrille betrachtet, entsteht der 3D-Eindruck.\par 
	Die Steuerung der Anwendung erfolgt über Tastatur und Maus bzw. Präsentationspointer.
	\subsection{Aufgabenstellung}
	Ziel des Projektseminars ist es, die Steuerung der Anwendung hinsichtlich einer Präsentation vor einer Zuschauergruppe zu erleichtern und intuitiv zu gestalten, sodass parallel an der Universität vorhandene (und bislang ungenutzte) Technik verwendet und präsentiert werden kann. In diesem Sinne geeignet und vorgeschlagen sind
	\begin{itemize}
		\item ein professionelles Trackingsystem zum Tracken von Raumpunkten und
		\item die Verwendung einer Microsoft Kinect 2 zur Gestenerkennung.
	\end{itemize}\par 
	Das damit entwickelte Programm soll Folgendes leisten:
	\begin{itemize}
		\item Es soll in der Lage zu sein, sämtliche Steuerung und Manipulation, die oben beschrieben wurde durchzuführen.
		\item Die Bedienung soll sehr intuitiv und einfach sein, d.\,h. etwaige Gesten müssen bezüglich der ihnen zugeordneten Aktion einleuchtend und leicht auszuführen sein.
		\item Das Programm soll möglichst einfach eingebunden und wiederverwendet werden können.		
	\end{itemize}