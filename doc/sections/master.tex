\subsection{Der Master}
	Der Master ist die Person (unter den getrackten Personen), der es obliegt, die Anwendung zu steuern, d.\,h. in unserem Anwendungsfall der Präsentation ist der Master der Präsentierende.
	Es muss gewährleistet werden, dass nur der Master das Programm steuert und dabei von weiteren Personen im Raum nicht (bzw. nicht ohne weiteres) gestört werden kann. Die Erkennung muss robust gegen Jittering der Kinectdaten sein.\par
\subsubsection{Möglichkeiten der Master-Identifikation}
	Grundsätzlich kamen für die Festlegung des Masters zwei Ideen auf. Zunächst sollte bei jedem Frame die getrackte Person identifiziert werden, die der Kinect bzgl. der z-Koordinate am nähesten ist und diese als Master festgelegt werden. Der Master könnte hierbei bei jedem Frame zwischen den getrackten Personen wechseln. \par
	Die zweite Möglichkeit war die Festlegung des Masters auf eine bestimmte Person, von der zunächst bestimmte Identifikations-Merkmale eingespeichert werden und die dann anhand dieser als Master reidentifiziert werden kann. Sofern diese Festlegung erst einmal geschehen ist, bleibt diese Person Master, selbst nachdem sich diese zwischenzeitlich in einem ungetrackten Zustand (beispielsweise beim Herausgehen aus dem getrackten Bereich) befunden hat und dann wieder als getrackt erkannt wird. Bei einer Recherche, welche Merkmale sich aus den von der Kinect gelieferten Daten extrahieren lassen ließen, um hierfür in Frage zu kommen, stießen wir hierbei auf verschiedene Möglichkeiten, von denen einige jedoch aufgrund ihrer Unpraktikabilität ausschieden (Erkennung anhand des Gangs oder anhand der Stimme würde bei unserer Anwendung keinen Sinn ergeben, da die Master-Person während der Bedienung kaum umher läuft und diese hierfür nicht zu sprechen braucht). Schließlich stießen wir auch auf Verfahren, die die Skelettdaten der Kinect zur Identifikation nutzen. Dies schien die für unsere Zwecke praktikabelste Lösung zu sein, wenngleich unser Ansatz die Skelettdaten zu nutzen im Vergleich zu denjenigen in den gefundenen Arbeiten stark vereinfacht wurde.\par
	Grundlegendstes Prinzip für die Identifizierung einer Person ist hierbei das Auslesen der Skelettkoordinatenpunkte mit Hilfe des Kinect SDKs und daraus der Ermittlung diverser Längen als Körperproportionen mittels der Berechnung des euklidischen Abstands zwischen den entsprechenden Skelettpunkten. Beispielsweise wird die rechte Oberarmlänge als Abstand zwischen dem rechten Schulterpunkt und dem rechten Ellenbogenpunkt ermittelt. Weitere Körperproportionen die verwendet wurden sind unter anderem die Schulterbreite, Hüftbreite, Unterarmlänge, Abstand zwischen Hals und Kopf. \par

\subsubsection{Robustheit}
Nach einigen Experimenten mit den Körperproportionen wurde festgestellt, dass einige Proportionen übermäßig große Abweichungen aufwiesen, wenn die gleiche Person in unterschiedlichen Posen mit der Kinect vermessen wurde bzw. dass sich einige Skelettpunkte \glqq{}verschieben\grqq{}, wenn man mit einem Körperteil darüber fährt. (Beispielsweise die gemessene Schulterbreite, welche sich signifikant zwischen der Standard-Pose und einer Pose bei der die Arme über den Kopf gestreckt sind unterschied.) 
\begin{figure}[h!]
		\centering
		\includegraphics[width=.8\textwidth]{pictures/standardpose_.png}
		\caption{Die Standardpose, in der sich eine Person befinden muss, damit sie vermessen werden kann.}\label{fig:standardp}
		\end{figure}
Aus diesem Grund wird die Vermessung einer Person nur durchgeführt, wenn sich die entsprechende Person über eine bestimmte Anzahl von Frames (etwa 20) durchgehend in der festgelegten Standard-Pose befindet. Falls während der Sammlung das Einhalten der Standard-Pose unterbrochen wird, wird die Sammlung erneut begonnen.\par
Ein weiteres Problem, das zu lösen war bestand darin, dass selbst beim Stillhalten, jedoch hauptsächlich bei der Bewegung einer getrackten Person, Skelettpunkte kurzzeitig auf einen anderen weiter entfernten Raumpunkt springen konnten, was teils zu unrealistischen Messungen einer Körperproportion für einzelne Frames führte (z.B. Messung einer Oberarmlänge von 2 Metern). Um derartige Ausreißer zu eliminieren wird der während der Vermessung einer Person zur Einspeicherung als Master über mehrere Frames der Mittelwert sowie die Standardabweichung jeder Körperproportion berechnet und jeder Wert der nicht innerhalb des Intervalls um den Mittelwert mit der Ausdehnung der Standardabweichung liegt, d.h. des Intervalls \[(\text{Mittelwert}-\text{Standardabweichung}\cdot \text{Faktor}, \text{Mittelwert}+\text{Standardabweichung}\cdot \text{Faktor})\text,\] ignoriert. 
\par
Weiterhin wird darauf geachtet, dass alle für die Vermessung relevanten Körperproportionen mit ausreichender Konfidenz von der Kinect getrackt werden. Die Kinect liefert hierfür für jeden Skelettpunkt einen von drei möglichen Konfidenzwerten, der angibt, wie wahrscheinlich die von der Kinect gelieferte Koordinatenwerte für diesen Punkt dem tatsächlichem Wert entsprechen. (Die drei Konfidenzwerte heißen Tracked, Inferred, NotTracked; wobei Tracked die höchste Konfidenz und NotTracked die niedrigste Konfidenz für einen Skelettpunkt bezeichnet). Ist in einem Frame der Konfidenzwert von einem der beiden Skelettpunkte, aus denen eine Körperproportion berechnet wird, nicht Tracked, so wird diese Körperproportion für diesen Frame nicht berücksichtigt bzw. mit Null gewichtet. \par
\subsubsection{Umsetzung}
Eingebettet in unsere Hauptschleife geht die Master-Identifikation folgendermaßen von statten: Beim Start des Programms ist zunächst einmal die Person Master, die bzgl. der Kinect die geringste z-Koordinate aufweist. Hierfür wird bei jedem Schleifendurchlauf abgefragt, welche der 6 (potenziell) getrackten Personen, den geringsten z-Wert hat. Der Master wird somit bei jedem Frame neu bestimmt.
\par
 Durch die Betätigung einer vorher festgelegten Taste wird schließlich die Master-Festlegung auf eine bestimmte Person aktiviert. Hierzu werden die Körperproportionen der Person, die als erstes in Standardpose erkannt wird, über mehrere Frames aus den Skelettdaten extrahiert, gepuffert und schließlich aus diesen für jedes Körpermerkmal der Mittelwert sowie die Standardabweichung berechnet. (Sofern beim Tastendruck noch keine Person im getrackten Bereich war, wird gewartet bis eine Person getrackt wird und diese die Standard-Pose einnimmt.) Der Mittelwert und die Standardabweichung werden dazu verwendet Ausreißer zu identifizieren und zu entfernen, um schließlich die Mittelwerte erneut ohne die Ausreißer zu berechnen. Neben diesen Mittelwerten für die nun als Master festgelegte Person, wird außerdem die von der Kinect vergebene ID der Person eingespeichert. Diese bleibt dieser Person zugeordnet, solange sie durchgehend getrackt wird. Verlässt der Master den getrackten Bereich oder wird anderweitig vorübergehend nicht getrackt (beispielsweise wenn dieser von einer anderen Person verdeckt wird) und kommt danach wieder in den getrackten Bereich zurück, hat dieser eine neue ID. In diesem Fall muss der Master neu identifiziert werden. 
 
 \begin{figure}
\includegraphics[width=\textwidth]{pictures/sensors-16-01965-g005.jpg}
\caption{Neuzuordnung von IDs nach zwischenzeitlichem \glqq Verlust\grqq{} von Skeletten, Quelle:\cite{bodyprop}}
\label{fig:fehlerk}
\end{figure}
 \par
 Solange der Master noch nicht gefunden wurde, werden hierfür für jede Person, die in Standard-Pose erkannt wird, die Körperproportionen über eine gewisse Anzahl (etwa 20) an Frames mit denen der für den Master eingespeicherten Werte verglichen. Schließlich wird aus den frameweise berechneten Abweichungen der Durchschnitt berechnet. Ist diese durchschnittliche Abweichung kleiner als ein im Programm festgelegter Wert ist diese als Master identifiziert worden. Diese kann nun mit der Steuerung des Programms fortfahren.
	
	