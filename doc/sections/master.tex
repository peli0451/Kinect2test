\subsection{Der Master}
	Der Master ist die Person (unter den getrackten Personen), der es obliegt, die Anwendung zu steuern, d.\,h. in unserem Anwendungsfall der Präsentation ist der Master der Präsentierende.\par
	Es muss gewährleistet werden, dass nur der Master das Programm steuert und dabei von weiteren Personen im Raum nicht (bzw. nicht ohne weiteres) gestört werden kann. Die Erkennung muss robust gegen Jittering der Kinectdaten sein.\par
	Grundsätzlich kamen für die Festlegung des Masters zwei Ideen auf. Zunächst sollte bei jedem Frame, die getrackte Person, identifiziert werden, die der Kinect bzgl. der z-Koordinate am nähesten ist und diese als Master festgelegt werden. Der Master könnte hierbei bei jedem Frame zwischen den getrackten Personen wechseln. \par
	Die zweite Möglichkeit war die Festlegung des Masters auf eine bestimmte Person, von der zunächst bestimmte Identifikations-Merkmale eingespeichert werden und die dann anhand dieser als Master reidentifiziert werden kann. Sofern diese Festlegung erst einmal geschehen ist, bleibt diese Person Master, selbst nachdem sich diese zwischenzeitlich in einem ungetrackten Zustand (beispielsweise beim Herausgehen aus dem getrackten Bereich) befunden hat und dann wieder als getrackt erkannt wird. Bei einer Recherche, welche Merkmale sich aus den von der Kinect gelieferten Daten extrahieren lassen ließen, um hierfür in Frage zu kommen, stießen wir hierbei auf verschiedene Möglichkeiten, von denen einige jedoch aufgrund ihrer Unpraktikabilität ausschieden (Erkennung anhand des Gangs oder anhand der Stimme würde bei unserer Anwendung keinen Sinn machen, da die Master-Person während der Bedienung kaum umher läuft und diese hierfür nicht zu sprechen braucht). Schließlich stiessen wir auch auf Verfahren die die Skelettdaten der Kinect zur Identifikation nutzen. Dies schien die für unsere Zwecke praktibalste Lösung zu sein, wenngleich unser Ansatz die Skelettdaten zu nutzen im Vergleich zu denjenigen in den gefundenen Arbeiten stark vereinfacht wurde.\par
	
	
	