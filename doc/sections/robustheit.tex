	\subsection{Robustheit und Pufferung}\label{sec:robustheit}
	Wie vorangegangen festgestellt wurde, sind einige der zu implementierenden Mechanismen anfällig gegenüber qualitativ niedrigwertigen Kinectdaten. Die fortwährende Auseinandersetzung mit den verschiedenen Gesten in verschiedenen Situationen lieferte in Verbindung mit passend gewählten Testszenarien eine Liste kritischer Punkte, die unten aufgeführt ist. Dabei war entscheidend, wie sich die Handhabung des Programmes subjektiv für den Benutzer anfühlt -- dies ist nicht gut mit Testdaten belegbar, sondern eher auf direktes Feedback des Steuernden angewiesen. Aus den so gewonnenen Erkenntnissen darüber, welche Kinect-Eigenschaften die Bedienbarkeit des Programmes aus Nutzersicht einschränken oder behindern, werden Mechanismen zu deren Behandlung entwickelt.
	\subsubsection{Auftretende Probleme}\label{sec:rob1}
	Nachfolgend sind zunächst die verschiedenen Probleme genannt. Dies sind Situationen und Szenarien, in denen die Kinect schlechtere Werte liefert. Es wird geschildert, wie der jeweilige Fehler entsteht und gegebenenfalls in welcher Hinsicht die resultierenden Daten schlecht sind.
	\begin{enumerate}[label=(\roman*)]
	\begin{figure}[H]
\includegraphics[width=\textwidth]{pictures/sensors-16-01965-g006.jpg}
\caption{Diverse falsch erkannte Skelette: degenerierte Skelette (a), verschmolzene Skelette (b) und Skelettverlust bei Überdeckung (c). Quelle: \cite{bodyprop}}
\label{fig:fehlerk}
\end{figure}
		\item\label{itm:problem1} \textbf{Gelenke in der Nähe von Objekten und anderen Körpern.} Diese können falsch oder verzerrt erkannt werden (siehe Abbildung \ref{fig:fehlerk}). So kann etwa die erkannte Handposition zwischen zwei Kinectframes Raumunterschiede von mehreren Metern aufweisen und zurückspringen. Dies kann auch Körperteile betreffen, die vor oder hinter anderen Körperteilen liegen. Für hinter Körpern oder Objekten versteckte Teile ist klar, dass diese von der Kinect nur geraten werden können. Gliedmaßen, die sich vor Körpern (seltener Gegenständen) befinden können durch die Kinect-Optik zum Teil nicht hinreichend von den weiter hinten befindlichen Körperregionen differenziert weren. Das Problem verschärft sich mit zunehmender optischer und räumlicher Ähnlichkeit, d.\,h. es ist insbesondere vom Rückstrahlverhalten im Infrarotlicht und der Position von Personen im Raum (nah beieinander, verdeckend oder verstreut) abhängig. Ferner steigt die Ungenauigkeit, je weiter getrackte Personen vom \glqq Abdeckbereich\grqq{} der Kinect (siehe \ref{sec:kinect}) entfernt sind.\par
\par 
	\begin{figure}[H]
	\centering
	\includegraphics[width=.7\textwidth]{pictures/irschatten.png}
	\caption{Im Sonderfall von Punkt eins beschriebene Situation eines Infrarotschattens nebst zugewiesenem (fehlerhaften) Skelett.}\label{fig:irsch}
	\end{figure}
		Ein Sonderfall fehlerhafter Skelette erwächst aus der Lichtabhängigkeit der Kinect in Verbindung mit dem von ihr verwendeten Verfahren der Tiefenfeststellung, im Falle der Kinect 2 das Time-Of-Flight-Verfahren, das in Abschnitt \ref{sec:kinect} skizziert wurde. Zur Illustration siehe Abbildung \ref{fig:irsch}. Die in den Raum gestrahlten Infrarotstrahlen können bei Spiegelungen unvorhergesehene Nebeneffekte bewirken. Nach Kenntnis dieser Problematik wurde diese Lichtabhängigkeit zusätzlich getestet, indem die Testperson Kleidung (hier Hosen) von hohem Infrarot-Rückstrahlvermögen trug. In dem in der Abbildung dargestellten Fall ist die Infrarotreflexion des Bodens und der Hose so groß, dass die Hose einen \glqq{}Infrarotglanzfleck\grqq{} vor die Beine des Probanden wirft. Dies ist problematisch, da über die Zeitmessung festgestellt wird, dass dieser Bodenbereich eine ähnliche Tiefe aufweist, wie der Körper der Testperson. Daher wurden dem Skelett des Probanden beim Test unnatürlich lange Beine zugewiesen (vgl. erneut Abbildung \ref{fig:irsch} am unteren Bildrand). Der Boden in der Testsituation war dabei eine diffus reflektierende Fläche ohne Ähnlichkeit zu einem idealen Spiegel.\par
		\item\label{itm:problem2} \textbf{Status der Hände.} Auch bei durchgängiger Aufrechterhaltung eines Handzustands kann es passieren, dass die Kinect vereinzelt falsche Zuweisungen trifft (eine offene Hand wird etwa als Lasso erkannt) oder keine Zuweisung möglich ist (Handstatus unbekannt, obwohl augenscheinlich einer der anderen Status zutrifft). Besonders schlecht wird die Erkennung, wenn sich die Hände vor dem Körper befinden. Sind die Hände selbst vollständig oder auch nur teilweise verdeckt, ist selbstverständlich ebenfalls keine sinnvolle Erkennung des Handstatus möglich. Hier liegen weitestgehend dieselben Mechanismen zugrunde, die auch die Probleme von Punkt eins verursachen.  
		\item\label{itm:problem3} \textbf{Jitter.} Die Kinectdaten sind verrauscht und weisen bspw. von Frame zu Frame kleine Ungenauigkeiten und Abweichungen der Gelenkpositionen in beliebige Richtungen auf. Diese Fehler nehmen ebenfalls umgekehrt proportional zur Kinect-Sicherheit zu, d.\,h. treten vermehrt auf, wenn die genaue Position nicht richtig erkannt wird, also besonders stark bei den Gelenken mit gefolgerten Positionsdaten (siehe Auflistung der Trackingstatus in \ref{sec:kinect}). Insbesondere ist dies wieder bei Verdeckung (egal in welcher Reihenfolge) mit infrarot"=optischer und räumlicher Nähe der Fall.\par 
		Die bereits in Punkt eins (Gelenke in der Nähe von Objekten / Personen) genannte Fehlersituation erzeugt teils ähnliche Effekte (ein \glqq{}Zittern\grqq{} von Gelenkpunkten in verschiedenste Richtungen). Der hier unter \glqq{}Jitterfehler\grqq{} aufgeführte Fehlerfall ist jedoch von Fehlern nach Punkt eins insofern abgegrenzt, dass hier es sich hierbei um ständig auftretende, von ihrer Natur her kleine Fehler handelt und die Probleme auch auftreten, wenn der betroffene Körper sich vollständig und unverdeckt im Kinect-Aufnahmebereich befindet.
				\begin{figure}[H]
	\centering
	\includegraphics[width=.6\textwidth]{pictures/ninja.png}
	\caption{Infrarotbild der Kinect mit Phantomskelett (oben links, magentafarben, entlang der Lampe erkannt). Insbesondere werden Teile des Phantoms von der Kinect mit \glqq{}hoher Konfidenz\grqq{} erkannt (verdeutlicht durch die dicken Linien, zu den \glqq{}Konfidenzstufen\grqq{} siehe \ref{sec:kinect}).\\
	Ferner verzerrtes Skelett am rechten Rand (verursacht durch Verlassen des Aufnahmebereichs).}\label{fig:ninja}
	\end{figure}
			\begin{figure}[H]
	\centering
	\includegraphics[width=.6\textwidth]{pictures/stuhlmensch.png}
	\caption{Infrarotbild der Kinect mit an Stühlen entstandenen Phantomskeletten.}\label{fig:stuhlmensch}
	\end{figure}
		\item\label{itm:problem4} \textbf{Phantome.} Mitunter kann es passieren, dass ein ganzes Skelett an einem Gegenstand \glqq{}hängenbleibt\grqq{}, siehe Abbildungen \ref{fig:ninja} und \ref{fig:stuhlmensch}. Dies geschieht, wenn eine getrackte Person einen Gegenstand passiert und dabei nicht korrekt erkannt wird -- etwa weil sich die Situation sehr nahe am Rand des Aufnahmebereichs abspielt oder der Gegenstand eine nahezu humanoide Form hat. Sobald die Person wieder richtig erkannt wird, erhält sie ein neues Skelett. Die Skelettdaten dieser Phantome  variieren über die Zeit sehr stark und willkürlich.\par
	\par
	\item\label{itm:problem5} \textbf{Skelettzuweisung.} Besonders problematisch für Mastererkennungsmechanismen ist ein anderer Fall von Fehlerskeletten, der von der Problematik aus Punkt eins zu unterscheiden ist: In dem Augenblick, in dem eine neue Person im Aufnahmebereich erkannt wird, bekommt sie ein Skelett zugewiesen. Wie diese Zuweisung genau stattfindet, ist nicht bekannt, vermutlich fließen jedoch nichtdeterministische Faktoren ein. Vorstellbar wäre dies als Identifizieren des Körpers mit einem geschätzten Skelett. Das Problem hierbei ist, dass eine erneute Skelettzuweisung bei zwischenzeitlichem Verlust anders ausfallen kann und der Körper somit leicht andere Proportionen hat. Derartiges lässt sich während der Anwendung ohne die visuellen Debug-Möglichkeiten des Kinect Studios nur schwer verhindern. Der Effekt scheint sich jedoch mildern zu lassen, wenn die betroffenen Personen sich bewegen und der Kinect damit mehr Schätzungsmöglichkeiten und die Möglichkeit zur Korrektur bieten.\par
	\end{enumerate}
\subsubsection{Auswirkungen aus Nutzersicht}
	Die in Abschnitt \ref{sec:rob1} genannten Punkte können gravierende Einschränkungen bezüglich der Programmbedienbarkeit mit sich ziehen.
	\begin{itemize}
	\item Eine fehlerhafte Erkennung von Positionen gemäß des \hyperref[itm:problem1]{ersten Punktes} (Situation: Gelenke in der Objekt- / Personennähe) kann zu einem gänzlichen Verlust der gegenwärtigen Position im virtuellen Raum führen: Im naiven Ansatz der Direktauswertung der Daten wird ein hoher Differenzwert zwischen den die Bewegung (oder Drehung) bestimmenden Handpositionen festgestellt, der die Stärke der Manipulation bestimmt und demzufolge auch eine extrem starke Manipulation bewirkt. Bei der Arbeit mit der Objektsteuerung ist ein weiteres Fehlerszenario aufgefallen, welches als Sonderfall dieses Punktes betrachtet werden kann: Ist die Hand des Nutzers, mit welcher dieser das Objekt bewegt oder dreht, durch die Kinect genau im Profil zu sehen, d.\,h. die Handfläche ist bezüglich der Höhenachse 90 Grad verdreht, so kann die Kinect nicht genau erkennen, in welche Richtung sie verdreht ist (m.\,a.\,W., ob sich der Daumen vorne oder hinten befindet). Solange dies der Fall ist, kann es passieren, dass die Kinect zwischen beiden Möglichkeiten hin- und herspringt, was sich in einer plötzlichen und äußerst starken Drehung widerspiegelt, der jedoch keinerlei bewusste Nutzereingabe zugrunde liegt.
	\item Das vorübergehende Verlieren (oder Missinterpretieren) des vorgeführten HandStates (\hyperref[itm:problem2]{Punkt zwei} von oben) äußert sich bei der Programmsteuerung dagegen in einem Stottern, d.\,h. dass die ursprünglich fortlaufend präsentierte Geste zu den Zeitpunkten der Fehlerkennung nicht wirkt und daher z.\,B. eine kontinuierlich angedachte Bewegung mehrfach abrupt unterbrochen wird. Siehe Abb. \ref{fig:fehlerk} für eine Illustration.
	\item Jitterfehler nach \hyperref[itm:problem3]{Punkt drei} verursachen durch die vielen willkürlichen kleinen Bewegungen eine als \glqq zittrig\grqq{} wahrgenommene Steuerung der Anwendung: So nimmt ein bewegtes Objekt etwa eine Vielzahl kleiner Bewegungen bzw. Drehungen in verschiedene Richtungen vor, ohne dass der Nutzer eine entsprechende Geste präsentiert hat.\par 
	\item Die Erzeugung eines Phantoms (\hyperref[itm:problem4]{Punkt vier}) ist vor allem dann kritisch, wenn das Phantom aus dem augenblicklichen Master hervorgeht. In diesem Falle übernimmt das Phantom die Programmsteuerung, wodurch einerseits der eigentliche Master die Kontrolle verliert, aber auch andererseits das Programm gänzlich chaotische Bewegungen und Drehungen vornehmen könnte. Letzteres ist wegen der Mechanismen zur Gestenerkennung jedoch unwahrscheinlich, da das Phantomskelett die entsprechenden, relativ strengen Constraints  erfüllen müsste, um den Idle-Modus zu verlassen, nach seiner Erzeugung das Programm aber  sehr schnell in den Idle-Modus bringt.\par 
	\item Schließlich ist klar, dass Fehler nach \hyperref[itm:problem5]{Punkt fünf} (Skelettneuzuweisung) die Mastererkennung erschweren, da der Master gegebenenfalls mit fehlerhaften Proportionen eingespeichert wird.
	\end{itemize}
\par\bigskip
	Diese Probleme üben einen negativen Einfluss auf die Erfahrung aus, die der Nutzer mit der Software macht. Insbesondere Fehler nach dem erstgenannten Schema können dem Nutzer das Erreichen seines Zieles -- etwa des Annavigierens eines Objektes -- unmöglich machen. Die weiteren Punkte werden dagegen einfach als störend empfunden.
	\subsubsection{Behandlung}
	Die verschiedenen (und auch üblichen) von angewendeten Mechanismen, um diese Probleme zu beheben, sollen hier erklärt werden. Ein Großteil der genannten Schwierigkeiten ergeben sich aus den üblichen Problemen von Sensoren (korrekte Messung physikalsicher Sachverhalte, hier zusätzlich mit Zeitanforderungen). Hinzu kommt, dass die Kinect als für ein breiteres Publikum ausgelegtes System über vergleichsweise preiswerte Sensoren verfügt (siehe \ref{sec:kinect}). In diversen Arbeiten, die sich mit ähnlichen Problemstellungen beschäftigen (s. \cite{biomid}, \cite{bodyprop}, \cite{kinectlight} und \cite{thermalsens}), finden die Grenzen der Kinect häufig Erwähnung und ein wesentlicher Punkt in der Auseinandersetzung mit der Kinect und ihrer Anwendung in dem gegebenen und ähnlichen Szenarien widmet sich einer möglichst fehlerarmen Auswertung der Daten. In der Regel wird dabei auf Verzerrungen und schlechte Werte eingegangen, die sich durch den eingeschränkten \glqq Abdeckbereich\grqq{} der Kinect und den Einfluss von Licht ergeben (siehe \cite{bodyprop} und \cite{kinectlight}). Ferner wird darauf hingewiesen, dass das Detektions- und Trackingproblem generell von Beleuchtung, Blickwinkel, Distanz und weiteren Faktoren abhängt (vgl. \cite{thermalsens}). Es ist jedoch zu beachten, dass sich die Betrachtungen in den diversen wissenschaftlichen Arbeiten z.\,T. noch auf die Kinect-Version 1 beziehen, die über teilweise andere Hardwarespezifikationen und Verfahrenstechniken verfügte (siehe \ref{sec:kinect}).\par
	Ferner ist ein gerade für das vorliegende Projekt wichtiger Punkt die Abhängigkeit der Kinect-Daten von der Pose, was etwa bereits in \cite{biomid} festgestellt wurde: Die Autoren haben dort Unterschiede bei Messungen verschiedener Posen festgestellt, die statistisch signifikant waren (s. ebd., S. 6). Generell ist es möglich, durch ungünstiges Verdecken von Körperpartien die durch die Kinect erkannten Gelenkpositionen zu verschieben. Im Test war es so möglich, eine Verschiebung des Genicks (bzw. des Gelenkpunktes zwischen Schultern und Kopf) um mehrere Zentimeter reproduzieren, indem die Hände vor dieser Stelle auf und ab bewegt werden. Besonders kritisch für Steuerungsaufgaben ist dies vor allem dann, wenn die Verdeckung nach dem Verschieben aufgehoben wird und der Gelenkpunkt an seine eigentliche Position \glqq zurückschnappt\grqq{}. Die zur Behandlung der diversen Fehler erfolgt nach zwei Grundmotiven. Dazu werden die Daten einerseits gemittelt und andererseits auf sinnvolle Bereiche beschränkt.
	\begin{itemize}
	\item Der ursprüngliche Ansatz, Pufferung und Mittelung, eliminiert die jitterartigen Fehler, mit denen die Kinect-Daten belastet sind. Hierzu wird ein Puffer vorher festgelegter Länge verwendet und während des Programmablaufs mit den für den Anwendungszweck wichtigen Daten, hier den Handpositionen des Nutzers gefüllt. Wenn schließlich die Rückgabeparameter für die Manipulationen bestimmt werden sollen, wird dieser Puffer ausgewertet. Dabei wird ein exponentiell gewichtetes Mittel der gepufferten Positionen gebildet. Die neuesten Puffereinträge werden am stärksten gewichtet. Die Pufferlänge wurde genau so angelegt, dass das dadurch erzeugte Delay dem Nutzer nicht unangenehm auffällt und gleichzeitig die Kontrolle über das Programm per Gestensteuerung wesentlich glatter und angenehmer erfolgen kann.
	\item Die eben beschriebene Glättung mag zwar kleine Jitterfehler behandeln, versagt jedoch bei Kinect-Daten, die sehr stark von den eigentlichen Realdaten abweichen (vgl. hierzu erneut die Auflistung von Problemsituationen aus Abschnitt \ref{sec:rob1}). Ein Beispiel für dieses immer wieder auftauchende Problem ist etwa ein weiterer Nutzer der sich im Hintergrund des steuernden Nutzers bewegt. In einem solchen Fall (und ähnlichen Fällen) kann es passieren, dass die Kinect Körperteile dieses zweiten Nutzers falsch interpretiert und dem Steuernden zuordnet. Dadurch können z.\,B. Positionsdaten entstehen, die um mehrere Meter von der Realtität abweichen. Diese Fehler benötigen eine eigene Ausreißerbehandlung: Werte, die eine zu große Abweichung von den zuletzt ermittelten Werten aufweisen (etwa eine Änderung der Handposition um mehrere Meter in aufeinanderfolgenden Frames) und daher unplausibel sind, werden auf eine vordefinierte Maximalabweichung abgeschnitten. Ohne eine solche Behandlung hätten diese Ausreißer dazu führen können, dass der Nutzer seine aktuelle Position in der 3D-Welt ohne sein Zutun mit großer Geschwindigkeit verlässt (falls er sich etwa im Kamerabewegungsmodus befand).\end{itemize}
	Die genannten Methoden bieten Robustheit, was fehlerhafte Kinect-Daten hinsichtlich der Position von Joints (Gelenkpunkten) angeht. Dies ist jedoch nicht der einzige Aspekt der Anwendung, der fehleranfällig ist und solche Sonderbehandlungen verlangt. Es wurde im Vorfeld ein weiterer solcher Punkt genannt -- die Erkennung der Handzustände. Der naive Vorgang, die Handzustände diskret zu erkennen und auszuwerten, würde bei der gegebenen Zustandsmashcine zu sofortigen Zustandswechseln führen. Eine Verbesserung dessen ist, statt der diskreten Gestenerkennung Fuzzy"=Gesten zu implementieren. Hierbei findet die Erkennung nicht nach dem Schema statt, dass eine konkrete Geste als erkannt zurückgegeben wird, sondern ein Tupel von Konfidenzwerten für die verschiedenen Gesten. Diese Konfidenzwerte ergeben sich aus dem Zustand, in dem sich die Zustandsmaschine befindet und den rohen Kinect-Daten für die Hände. So ist beispielsweise das folgende Szenario denkbar: Der Benutzer ist dabei, die Kamera zu verschieben, als eine seiner Hände fehlerhaft als geschlossen erkannt wird. Diese \glqq{}Geste\grqq{} ist so nicht definiert und mehrdeutig, es kann jedoch aufgrund des State-Machine-Zustands davon ausgegangen werden, dass der Nurzer wahrscheinlich weiterhin dabei ist, die Kamera zu verschieben -- es wird jedoch auch eine Restwahrscheinlichkeit für die Rückkehr in den IDLE-Zustand eingeräumt.\par 
	Die genannten Konfidenzwerte addieren sich über das Tupel zu 1 und können als Maß für das Vertrauen darin gesehen werden, dass die jeweilige Geste präsentiert wurde. Dies wird in Verbindung mit einem Puffer verwendet, der nach dem selben Prinzip wie oben funktioniert und kurzzeitige Hand-State-Fehler entfernen soll. Der Puffer wird genau wie oben ausgewertet (gewichtete Mittelwertbildung) und das erhaltene Mittelwerttupel gibt entsprechend seines Maximaleintrags die erkannte Geste wieder, die dann einen Zustandswechsel bewirken kann.\par\bigskip
	%
	Abschließend sei noch ein Szenario genannt, gegen das die Robustheitsmechanismen keinen hinreichenden Schutz bieten: Das der gezielten Manipulation. Wie bereits bemerkt, sind die Kinectdaten z.\,T. ungenau, etwa bezüglich der Gelenkpositionen. Durch (gegebenenfalls bewusstes) Verdecken oder Unkenntlichmachen von Körperteilen ist es möglich, die von der Kinect erkannten Jointkoordinaten zu verschieben. Dies kann durch den Einsatz von der Hände und der Körperhaltung, aber auch z.\,B. durch weite Kleidung hervorgerufen werden. Ein Beispiel ist die im Rahmen der Mastererkennung verwendete Torsolänge: Ein Nutzer, der von der Kinect getrackt wird, kann die an ihm erkannte Torsolänge (genauer den Abstand zwischen den entsprechenden erkannten Gelenkpunkten) verändern, indem er sich beispielsweise streckt und \glqq groß macht\grqq{} (dies verlängert die erkannte Torsolänge) oder aber sich leicht nach vorne beugt (was die erkannte Torsolänge staucht). Dies eröffnet ihm einen Spielraum bezüglich des genannten Merkmals, in dem er die vom eingespeicherten Master bekannte Torsolänge annähern kann. Ähnliches ist für andere Körperpartien reproduzierbar, etwa durch leichtes Beugen der Arme oder Anheben und Hängenlassen der Schultern. Eine weitere, aber weniger relevante Möglichkeit ist auch das Ausnutzen von Kinect-Ungenauigkeiten am Rande ihres Aufnahmebereiches, z.\,B. weit weg von der Kamera. Ihre kleinere Relevanz liegt in der Schwierigkeit begründet, \emph{bewusst} diverse Effekte hervorzurufen, da die Randungenauigkeiten aus Anwendersicht willkürlich und ohne Muster sind.\par\bigskip
	%
	Für Nutzer, die sich ohnehin aufgrund ihrer Körpermerkmale recht ähnlich sind, ist es bei solchen wie eben beschriebenen Manipulation schließlich nicht mehr sicher möglich, eine korrekte Entscheidung zu fällen. Es sei jedoch noch einmal darauf hingewiesen, dass der Beeinflussungsspielraum relativ gering und damit nur für a priori ähnliche Skelette von Belang ist. Ferner ist es augenscheinlich unmöglich, die Manipulation ohne Debugausgaben der genauen Werte bewusst und gezielt durchzuführen.
