	\subsection{Robustheit und Pufferung}\label{sec:robustheit}
	Wie wir vorangegangen festgestellt haben, sind einige der Mechanismen, die wir implementieren wollen anfällig gegenüber qualitativ niedrigwertigen Kinectdaten. Tests mit der Kinect haben folgende kritische Situtationen ergeben:
	\begin{itemize}
		\item Gelenke und Skelettbestandteile in der Nähe von Objekten und anderen Personen. Diese können falsch oder verzerrt erkannt werden. So kann etwa die erkannte Handposition zwischen zwei Kinectframes Raumunterschiede von mehreren Metern aufweisen und zurückspringen.
		\item Status der Hände. Auch bei durchgängiger Aufrechterhaltung eines Handzustands kann es passieren, dass die Kinect vereinzelt falsche Zuweisungen trifft oder keine Zuweisung möglich ist.
	\end{itemize}
	Beide Situationen lassen sich behandeln, indem Entscheidungen unseres Programms, nicht nur vom augenblicklichen Rückgabewert der Kinect abhängen, sondern auch einige vergangene Werte mit einbeziehen. So kann ermittelt werden, ob der aktuelle Wert (mit hoher Wahrscheinlichkeit) ein zu ignorierender Ausreißer ist. Dazu wird ein Ringpuffer verwendet und an den entsprechenden Stellen im Quellcode ein gewichtetes Mittel über den Pufferinhalt gebildet, wobei neuere Einträge mit deutlich größerem Gewicht eingehen. Für Gesten kann dann mit einer bestimmten Zuverlässigkeit eine Zuordnung getroffen werden, für Raumpositionen stellt dieses Mittel eine Glättung dar. Dies hat den positiven Nebeneffekt, dass die endgültige Anwendung der errechneten Parameter auf die Kamera bzw. das Objekt ebenfalls geschmeidiger werden.