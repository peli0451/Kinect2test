\section{Schlussbemerkungen}
In diesem Abschnitt wollen wir zum Abschluss einserseits feststellen, wie gut die im Projektseminar erstellte Anwendung den für sie vorgesehenen Zweck erfüllt, andererseits aber auch bemerken, wie sich die Kinect selbst überhaupt für eine solche Aufgabe eignet.\par 
Wir geben Ausblick auf sinnvolle und nützliche Erweiterungen der Software und schließen mit einer Gesamtbewertung des Projektes aus unserer Sicht.
%
%
\subsection{Eignung unserer Software}
Die im Projektseminar erstellte Software erfüllt die Aufgabenstellung in einem zufriedenstellenden Maße. Es ist möglich, die 3D-Szene und enthaltene Objekte per Geste zu steuern und diese Steuerung ist (in den Augen der Projektteilnehmer) intuitiv und technisch einfach durchzuführen. Die Steuerung arbeitet relativ präzise, zeigt jedoch in Teilen auch die Störanfälligkeit der Kinect auf. Bei schlechten Daten kann so dennoch ein für den Anwender ärgerliches Zittern oder Wegspringen auftreten. Wir halten den Nutzer daher dazu an, der Kinect im Rahmen seiner Möglichkeiten entgegenzukommen, d.\,h. zum Beispiel keine zu weite Kleidung (offene Jacken) zu tragen und die Hände bei der Steuerung vorzugsweise neben (nicht vor) dem Körper zu haben. Bei der einhändige Objektrotation ist der Einfluss der Kinectgüte am besten erkennbar, vor allem beim Kippen nach vorne bzw. hinten.\par 
Die Mastererkennung funktioniert ebenfalls wie vorgesehen, es kann jedoch im ungünstigsten Falle passieren, dass für den Master ein fehlerhaftes Skelett mit falschen Proportionen beobachtet wird. Dann wird er gegebenenfalls nicht wiedererkannt und das Programm muss neu gestartet werden, um wieder steuerbar zu sein.
%
\subsection{Eignung der Kinect}
Die Kinect eignet sich nur mit Einschränkung für eine sehr präzise $1$"=zu"=$1$"=Abbildung einer Steuerung. Man darf dabei jedoch nicht außer Acht lassen, welcher Herkunft sie ist und dass eine solche Verwendung nicht ihrem vorgesehenen Anwendungsspektrum angehört. Es wäre definitiv einfacher, eine Steuerung durch diskrete Gestenerkennung ohne $1$"=zu"=$1$"=Zusammenhang zu gestalten, bspw. so, dass eine bestimmte Geste eine Bewegung vordefinierter Distanz in eine der Achsenrichtungen bewirkt. Demgegenüber treten bei der Direktabbildung Probleme mit den qualtitativ (wenigstens im Vergleich zu industriell verwendeter Sensortechnik) schlechten bis sehr schlechten Kinectdaten auf. Trotz starker Glättung ist es uns nicht gelungen, den Jitter unter Erhalt der Echtzeitabbildung vollständig zu eliminieren. Auch mit Parameteroptimierung ist an dieser Stelle vermutlich nicht mehr viel zu erreichen.\par 
Bei der Mastererkennung ist die größte Problematik die im vorangegangenen Abschnitt genannte. Dazu kommt, dass es mit unserer Methode schwierig sein könnte, sehr ähnliche Personen zu differenzieren -- wobei bessere Systeme z.\,B. bei der Präsentation eineiiger Zwillinge vermutlich auch an ihre Grenzen kommen. Hierzu hatten wir doch keinen Zugriff auf passende Testpersonen. Mit größerem Aufwand lassen sich jedoch bessere Erkennungsquoten erzielen, vergleiche dazu etwa \cite{appearance}, \cite{biomid} und \cite{bodyprop}. Es gibt auch gänzlich andere Ansätze, wie etwa über den Gang (siehe \cite{gait}) oder das Gesicht (siehe \cite{face}).Selbstverständlich sind die sichersten Erkennungen nicht mehr echtzeitfähig.\par 
Wir schließen an dieser Stelle damit, dass die Kinect für den gegebenen Zweck verwendet werden kann, aber in ihrer Qualität nicht ohne Weiteres an Hochleistungssysteme heranreicht. Gerade zu sicherheitsrelevanten Aufgaben sollte man also nicht am Trackingsystem sparen.
%
\subsection{Verbesserungen}
Obwohl die Kernaufgabenstellung hinlänglich erfüllt ist, möchten wir Ausblick auf potenzielle Verbesserungen unserer Software geben.\par
Zunächst wäre es in der Zukunft natürlich wünschenswert, wenn sämtliche Stubfunktiononalität und der geplante Funktionsumfang implementiert würden, sobald das einbindende Grundprogramm sie unterstützt.\par 
Die Parameter des Programms ließen sich in der Zukunft weiter optimieren. Eventuell sind dafür auch Tests in einem größerem Rahmen -- z.\,B. mit etwa 50 Testpersonen -- nützlich. Diese würden den zusätzlichen Zweck erfüllen, einen besseren Einblick in die Unterschiede zwischen verschiedenen Skeletten zu gewinnen und so bspw. neue Körpermerkmale für die Erkennung zu gewinnen oder weniger aussagekräftige auszusondern.\par 
In der Hinsicht auf Programmparameter währen für den Anwender auch zusätzliche Schnittstellen nach außen wünschenswert, z.\,B. um die Genauigkeit der Mastererkennung von außen vorgeben zu können. Gerade vor diesem Hintergrund war während der Arbeit im Gespräch die Mastererkennung so zu gestalten, dass die Einspeicherung so lange fortwährt, bis die Standardabweichung aller gesammelten Daten unter eine vorgegebene Konvergenzschranke gerät und die erlaubte Abweichung damit anstelle der empirischen Feststellung präzise vorgegeben werden kann. Diese Stoßrichtung wurde aus Zeitgründen in der Arbeitsphase des Projektes verworfen. Insbesondere wäre dies nicht ohne Feedback für den Nutzer sinnvoll, da dieser sonst über keinerlei Wissen verfügt, wie lange die Masterfestlegung (noch) dauern wird. Daher sollte vor dieser Implementierung ein komplett funktionstüchtiges Eventsystem bereitstehen und von der Grundanwendung eine Ausgabemöglichkeit gegeben sein.\par 
Darüber hinaus gibt es Erweiterungen der Funktionalität, die die Bedienerfahrung verbessern würden. Dies betrifft etwa ein Zurücksetzen des Masters, um die Steuerung abgeben zu können; ein Zurücksetzen von Kamera und Objekten, ggf. per Geste; die Skalierung und das Löschen von Objekten mittels Gesten und sicher viele weitere. Einige solcher Wunschfunktionalitäten ergeben sich erst im Laufe der tatsächlichen Verwendung der Software, wie es etwa bei der hinzugekommenen Aufgabe, einen Flugmodus zu implementieren der Fall war.
%
\subsection{Fazit}
Trotz der Probleme der Kinect, sind wir davon überzeugt, dass unser Programm mit vertretbarem Aufwand eine verwertbare Softwarelösung der Aufgabenstellung darstellt und sie insbesondere für Präsentationen wie an einem Tag der offenen Tür einen Mehrwert darstellt.\par 
Sofern hohe Genauigkeit notwendig ist oder die Anwendung irgendwelchen Sicherheitskriterien genügen muss, empfehlen wir jedoch die Nutzung eines höherwertigen Trackingsystems zuungunsten der Kinect.\par 
Die Arbeit mit der Kinect und mit den Konzepten Gestensteuerung und Personenerkennung war für alle Projektteilnehmer eine wertvolle Erfahrung und abseits der üblichen Schwierigkeiten, die eine solche Projektsituation mit sich zieht, ein reizvolles und erfüllendes Unterfangen.\par\bigskip
Wir möchten abschließend dem Fachgebiet Graphische Datenverarbeitung für die zur Verfügung gesellte Technik nebst Leihmöglichkeit danken und insbesondere unseren Betreuer, Herrn M.\,sc. Julian Meder hervorheben, der zahlreiche Gedanken und Anregungen in das Projekt einfließen lies, kreativ in seinen Vorstellungen vom Funktionsumfang des Programmes war und unsere Fragen stets zügig und kompetent beantwortete.