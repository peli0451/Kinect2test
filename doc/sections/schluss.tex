\section{Schlussbemerkungen}\label{sec:schluss}
In diesem Abschnitt soll diskutiert werden, in welchem Umfang die entstandene Anwendung die an sie gestellten Anforderungen erfüllt und andererseits bemerken, wie die Kinect selbst überhaupt für eine solche Aufgabe geeignet ist.
%
%
\subsection{Eignung der Software}\label{sec:programm}
Die erstellte Software erfüllt die Aufgabenstellung in einem zufriedenstellenden Maße: Es ist möglich, die 3D-Szene und enthaltene Objekte per Geste zu Steuerung und diese Steuerung technisch einfach durchzuführen. Nach Rückkopplung aus diversen Selbstversuchen hat sich gezeigt, dass das Steuerempfinden für den Nutzer in der subjektiven Wahrnehmung in einem guten Maße präzise ist, was die Übertragung der Handbewegungen auf Objekte oder die Kamera betrifft. Bei schlechten Daten (siehe Abschitt \ref{sec:rob1}) kann es dennoch passieren, dass die Steuerung zittert oder springt. Um dies zu minimieren sollte zusätzlich auf Kooperation des Nutzers gesetzt werden in dem Sinne, dass er z.\,B. nicht zu weite Kleidung wie offene Jacken trägt und die Hände bei der Steuerung vorzugsweise neben (und nicht vor) dem Körper hat. Dies kommt dann der Kinect bei der Erkennung entgegen. Bei der einhändigen Objektrotation ist der Einfluss der Kinectgüte am stärksten sprbar, vor allem beim Kippen nach vorne oder hinten.\par
Die Mastererkennung funktioniert ebenfalls wie vorgesehen, es kann jedoch im ungünstigsten Falle passieren, dass für den Master ein fehlerhaftes Skelett mit falschen Proportionen beobachtet wird. Dann wird er gegebenenfalls nicht wiedererkannt und neu eingespeichert werden muss.
%
\subsection{Eignung der Kinect}
Wie sich durch das Projekt herausgestellt hat, ist die Kinect mit einigen Abstrichen bei der Datenqualität für $1$"=zu"=$1$"=Abbildungen geeignet. Demgegenüber steht aber ihre Verfügbarkeit, der einfache Zugang und der Preis (vgl. wieder \ref{sec:kinect}).\par 
Rückblickend wäre eine Steuerung durch diskrete Gestenerkennung ohne $1$"=zu"=$1$"=Zusammenhang einfacher zu implementieren gewesen, bspw. so, dass eine bestimmte Geste eine Bewegung vordefinierter Distanz in eine der Achsenrichtungen bewirkt. Dies würde zwar viele der Probleme aus Abschnitt \ref{sec:robustheit} vermeiden, trotzdem bleibt eine Direktabbildung wegen der größeren Freiheiten angenehmer. Außerdem gibt es, während die Bewegung einfach zu berwerkstelligen wäre, keinen natürlichen Ansatz einer solchen Gestenfindung für eine Rotation im Raum.\par 
Trotz starker Glättung ist es nicht gelungen, den Jitter unter Erhalt der Echtzeitabbildung vollständig zu eliminieren, sodass in Resten noch für den Nutzer sichtbar ist, wenn auch nur in einem subtilen Maße.\par 
Bei der Mastererkennung ist die größte Problematik die im vorangegangenen Abschnitt \ref{sec:programm} genannte. Dazu kommt, dass es mit den hier entwickelten Methode schwierig ist, sehr ähnliche Personen zu differenzieren -- wobei bessere Systeme z.\,B. bei der Präsentation eineiiger Zwillinge vermutlich auch an ihre Grenzen kommen. Hierzu bestand jedoch kein Zugriff auf passende Testpersonen. Mit größerem Aufwand lassen sich jedoch bessere Erkennungsquoten erzielen, vergleiche dazu etwa \cite{appearance}, \cite{biomid} und \cite{bodyprop}. Es gibt auch gänzlich andere Ansätze, wie etwa über den Gang (siehe \cite{gait}) oder das Gesicht (siehe \cite{face}). In einem gewissen Rahmen findet dabei stets ein Abtausch zwischen Sicherheit der Erkennung und Laufzeit statt, der je nach Anwendung zu berücksichtigen ist.\par 
An dieser Stelle sei damit geschlossen, dass die Kinect für den gegebenen Zweck verwendet werden kann, aber einiges an Nachbehandlung der Daten erfordert. Zu sicherheitsrelevanten Zwecken sollte der Einsatz höherwertiger Systeme in Betracht gezogen werden.
%
\subsection{Verbesserungen}
Die entstandene Software wird den Anforderungen aus Abschnitt \ref{sec:aufg} und diversen Erweiterungen gerecht. Dennoch soll hier Ausblick auf potenzielle Verbesserungen gegeben werden.
Zunächst wäre es in der Zukunft natürlich wünschenswert, wenn sämtliche Stubfunktiononalität und der geplante Funktionsumfang implementiert würden, sobald das einbindende Grundprogramm sie unterstützt.\par 
Die Parameter des Programms ließen sich in der Zukunft weiter optimieren. Eventuell sind dafür auch Tests in einem größerem Rahmen -- z.\,B. mit etwa 50 Testpersonen -- nützlich. Diese würden den zusätzlichen Zweck erfüllen, einen besseren Einblick in die Unterschiede zwischen verschiedenen Skeletten zu gewinnen und so bspw. neue Körpermerkmale für die Erkennung zu gewinnen oder weniger aussagekräftige auszusondern. Darüber hinaus bieten derartige Tests noch mehr individuelle Erfahrungsberichte, die genutzt werden können, um die Intuitivität und das Eintauchen des Nutzers in die virtuelle Umgebung zu verbessern.\par 
In der Hinsicht auf Programmparameter wären für den Anwender auch zusätzliche Schnittstellen nach außen wünschenswert, z.\,B. um die Genauigkeit der Mastererkennung von außen vorgeben zu können. Während der Arbeit war hier etwa im Gespräch, die Einspeicherung so lange fortzusetzen, bis die Standardabweichung aller gesammelten Daten unter eine vorgegebene Konvergenzschranke gerät und die erlaubte Abweichung damit anstelle der empirischen Feststellung präzise vorgegeben werden kann. Dies ist jedoch ohne Feedback für den Nutzer nicht sinnvoll möglich, da dieser sonst über keinerlei Wissen verfügt, wie lange die Masterfestlegung (noch) dauern wird. Daher sollte im Vorhinein ein komplett funktionstüchtiges Eventsystem und Möglichkeiten der Ausgabe in der Grundanwendung bereitstehen. Weiterhin löst dies nicht das Problem schlechter Skelettzuweisungen und es kann weiterhin passieren, dass Skelette mit falschen Proportionen eingespeichert werden. Zusammenfassend war hier das erwartete Aufwand-Nutzen-Verhältnis zu niedrig und es wurde nicht zuletzt aus Zeitgründen auf eine Implementierung verzichtet.
Darüber hinaus gibt es Erweiterungen der Funktionalität, die die Bedienerfahrung verbessern würden. Dies betrifft etwa ein Zurücksetzen von Kamera und Objekten, ggf. per Geste; die Skalierung und das Löschen von Objekten mittels Gesten und sicher viele weitere. Einige solcher Wunschfunktionalitäten ergeben sich erst im Laufe der tatsächlichen Verwendung der Software, wie es etwa bei der hinzugekommenen Aufgabe, einen Flugmodus zu implementieren der Fall war.
%
\subsection{Fazit}
Trotz der Probleme der Kinect stellt das hier entwickelte Programm mit vertretbarem Aufwand eine verwertbare Softwarelösung der Aufgabenstellung und insbesondere für Präsentationen wie an einem Tag der offenen Tür einen Mehrwert dar.\par 
Sofern hohe Genauigkeit notwendig ist oder die Anwendung irgendwelchen Sicherheitskriterien genügen muss, sei jedoch die Nutzung eines höherwertigen Trackingsystems zuungunsten der Kinect empfohlen.\par 